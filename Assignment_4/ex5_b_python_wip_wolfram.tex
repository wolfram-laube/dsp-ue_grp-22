\item[(b)]
\section*{Task (b)}

\subsection*{Problem Statement}
Generate a Hamming window of length \( N \) using the MATLAB command `hamming`. Multiply the Hamming window with the signal \( x \) to obtain the signal \( y \). Compute the spectrum of \( y \) and display its magnitude like in (a).

\subsection*{Python Script}
\begin{verbatim}
import numpy as np
import matplotlib.pyplot as plt

# Parameters
N = 128
n = np.arange(N)
w1 = 2 * np.pi * 0.1
w2 = 2 * np.pi * 0.15

# Generate the signal
x = np.cos(w1 * n) + np.cos(w2 * n)

# Generate the Hamming window
hamming_window = np.hamming(N)

# Multiply the Hamming window with the signal x to obtain y
y = x * hamming_window

# Compute the FFT of y
Y = np.fft.fft(y)

# Compute the magnitude of the spectrum
magnitude_spectrum_y = np.abs(Y)

# Generate frequency axis
frequencies = np.fft.fftfreq(N, d=1.0)

# Plot the magnitude spectrum of y
plt.figure(figsize=(10, 6))
plt.stem(frequencies, magnitude_spectrum_y, 'b', markerfmt=" ", basefmt="-b")
plt.title('Magnitude Spectrum of y (Hamming Window Applied)')
plt.xlabel('Frequency (Hz)')
plt.ylabel('Magnitude')
plt.grid(True)
plt.show()

# Save the plot for LaTeX inclusion
plt.savefig('fig/ex5_b_magnitude_spectrum_hamming.png')
\end{verbatim}

\subsection*{Magnitude Spectrum of the Signal with Hamming Window}
\begin{figure}[h]
    \centering
    \includegraphics[width=0.8\textwidth]{fig/ex5_b_magnitude_spectrum_hamming.png}
    \caption{Magnitude Spectrum of the Signal with Hamming Window Applied}
    \label{fig:ex5_b_magnitude_spectrum_hamming}
\end{figure}

\subsection*{Analysis}
The magnitude spectrum plot shows the discrete frequencies present in the signal after applying the Hamming window. The Hamming window reduces spectral leakage, resulting in a clearer representation of the frequency components compared to the rectangular window.
