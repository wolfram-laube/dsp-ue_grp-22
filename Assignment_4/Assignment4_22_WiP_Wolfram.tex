\documentclass[12pt,a4paper,austrian]{article}
\usepackage{graphicx}
\usepackage[austrian, english]{babel}
\usepackage[utf8]{inputenc}
\usepackage{listings}
\usepackage{multirow}
\usepackage{epstopdf}
\usepackage{amsmath}
\usepackage{amssymb}
\usepackage{hyperref} % fuer Mengen \N, Q, C, R
\usepackage{minted} % fuer Code Listings mit Syntax Highlighting
\graphicspath{{./fig/}}


%% Satzspiegel
\setlength{\hoffset}{-1in} \setlength{\textwidth}{18cm}
\setlength{\oddsidemargin}{1.5cm}
\setlength{\evensidemargin}{1.5cm}
\setlength{\marginparsep}{0.7em}
\setlength{\marginparwidth}{0.5cm}

\setlength{\voffset}{-1.9in}
\setlength{\headheight}{12pt}
\setlength{\topmargin}{2.6cm}
\addtolength{\topmargin}{-\headheight}
\setlength{\headsep}{3.5cm}
\addtolength{\headsep}{-\topmargin}
\addtolength{\headsep}{-\headheight}
\setlength{\textheight}{27cm}

%% How should floats be treated?
\setlength{\floatsep}{12 pt plus 0 pt minus 8 pt}
\setlength{\textfloatsep}{12 pt plus 0pt minus 8 pt}
\setlength{\intextsep}{12 pt plus 0pt minus 8 pt}

\tolerance2000
\emergencystretch20pt

%% Text appearence
% English text
\newcommand{\eg}[1]%
{\selectlanguage{english}\textit{#1}\selectlanguage{austrian}}

\newcommand{\filename}[1]
{\begin{small}\texttt{#1}\end{small}}

\newcommand\IFT{\unitlength1mm\begin{picture}(10,2) \put (1,1)
{\circle{1.7}} \put(2,1){\line(1,0){5}} \put(8,1)
{\circle*{1.7}}\end{picture}}
\newcommand\FT{\unitlength1mm\begin{picture}(10,2) \put (1,1)
{\circle*{1.7}} \put(2,1){\line(1,0){5}} \put(8,1)
{\circle{1.7}}\end{picture}}

% A box for multiple choice problems
\newcommand{\choicebox}{\fbox{\rule{0pt}{0.5ex}\rule{0.5ex}{0pt}}}

\newenvironment{wahrfalsch}%
{\bigskip\par\noindent\makebox[1cm][c]{richtig}\hspace{3mm}\makebox[1cm][c]{falsch}
    \begin{list}%
    {\makebox[1cm][c]{\choicebox}\hspace{3mm}\makebox[1cm][c]{\choicebox}}%
    {\setlength{\labelwidth}{2.31 cm}\setlength{\labelsep}{3mm}
    \setlength{\leftmargin}{2.61 cm}\setlength{\listparindent}{0pt}
    \setlength{\itemindent}{0pt}}%
    }
    {\end{list}}

\newcounter{theaufgabe}\setcounter{theaufgabe}{1}
\newenvironment{aufgabe}[1]%
{\bigskip\par\noindent\begin{nopagebreak}
                          \textsf{\textbf{\arabic{theaufgabe}.\thinspace Aufgabe}}\quad
                          \textsf{\textit{#1}}\\*[1ex]%
                          \stepcounter{theaufgabe}\hspace{2ex}
\end{nopagebreak}}
{\par\pagebreak[2]}

% Innerhalb der Aufgaben erfolgt die weitere Unterteilung mittels einer
% enumerate Umgebung, die allerdings a), b),... zaehlen soll.
\renewcommand{\labelenumi}{\alph{enumi})}
\renewcommand{\labelenumii}{\arabic{enumii})}

% A box to tick for everything which has to done
\newcommand{\abgabe}{\marginpar{$\Box$}}
% Margin paragraphs on the left side
\reversemarginpar

% Language for listings
%\lstset{language=Vhdl,
%  basicstyle=\small\tt,
% keywordstyle=\tt\bf,
% commentstyle=\sl}

% No indention
\setlength{\parindent}{0.0cm}
% Don't number sections
\setcounter{secnumdepth}{0}


%% Beginning of the text

\begin{document}
    \selectlanguage{austrian}
    \pagestyle{plain}


%===  This is the header section ============================================================
    \thispagestyle{empty}
    \noindent
    \begin{minipage}[b][4cm]{1.0\textwidth}
        \begin{center}
            \begin{bf}
                \begin{large}
                    Digital Signal Processing SS 2024 -- 4.~Assignment
                \end{large} \\
                \vspace{0.3cm}
                \begin{Large}
                    Reconstruction, DFT, FFT, STFT
                \end{Large} \\
                \vspace{0.3cm}
            \end{bf}
            \begin{large}
                Group 22\\
                Julian Feichtinger, K12015812\\
                Wolfram Laube, K08900915\\
            \end{large}
        \end{center}
    \end{minipage}

    \noindent \rule[0.8em]{\textwidth}{0.12mm}\\[-0.5em]
%=======================================================================================


    \begin{aufgabe}{Reconstruction (15\%)}

        Consider the analog signal

        $$
        x(t)=1+0.5 \cos \left(2 \pi f_{1} t\right)+2 \sin \left(2 \pi f_{2} t\right)+\sin \left(2 \pi f_{3} t\right)
        $$

        with $f_{1}=2 k \mathrm{kz}, f_{2}=4 \mathrm{kHz}$ and $f_{3}=6 \mathrm{kHz}$.

        \begin{enumerate}
            \item[a)] Sketch the Fourier transform of $x(t)$ and plot the analog signal $x(t)$ in Matlab using a timevector $t=0: 1 e-6: 1 e-3$.

            \item[b)]  Sample the analog signal $x(t)$ with sampling frequencies $f_{s 1}=9 \mathrm{kHz}$ and $f_{s 2}=14 \mathrm{kHz}$, which yield $x_{1}[n]$ and $x_{2}[n]$, respectively.
            Sketch the corresponding DTFT spectra.

            \item[c)]  After ideal reconstruction you end up with the analog signals $x_{1}(t)$ and $x_{2}(t)$.
            Sketch the reconstructed analog spectra.
            Plot the reconstructed time-domain signal.
            Compare your results with point a).
        \end{enumerate}

        \hrule
        \begin{enumerate}
            \item[(a)]
\section*{Task (a)}

\subsection*{Problem Statement}
Consider the analog signal
\[ x(t) = 1 + 0.5 \cos \left(2 \pi f_{1} t \right) + 2 \sin \left(2 \pi f_{2} t \right) + \sin \left(2 \pi f_{3} t \right) \]
with \( f_{1} = 2 \, \text{kHz} \), \( f_{2} = 4 \, \text{kHz} \), and \( f_{3} = 6 \, \text{kHz} \).

a) Sketch the Fourier transform of \( x(t) \) and plot the analog signal \( x(t) \) in Python using a time vector \( t = 0:1 \times 10^{-6}:1 \times 10^{-3} \).

\subsection*{Theoretical Background}
The Fourier transform is a mathematical operation that transforms a time-domain signal into its frequency-domain representation. It is crucial for analyzing the frequency components of signals.

Given an analog signal \( x(t) \), its continuous Fourier transform \( X(f) \) is defined as:
\[ X(f) = \int_{-\infty}^{\infty} x(t) e^{-j 2 \pi f t} \, dt \]

For a signal consisting of sinusoids, the Fourier transform will show spikes at the corresponding frequencies.

\subsection*{Mathematical Derivation}
The given signal is:
\[ x(t) = 1 + 0.5 \cos \left(2 \pi f_{1} t \right) + 2 \sin \left(2 \pi f_{2} t \right) + \sin \left(2 \pi f_{3} t \right) \]

The Fourier transform of this signal consists of delta functions at the frequencies of the cosine and sine terms:
\[ X(f) = \delta(f) + 0.25 \left[ \delta(f - f_1) + \delta(f + f_1) \right] - j \left[ \delta(f - f_2) - \delta(f + f_2) \right] - 0.5 j \left[ \delta(f - f_3) - \delta(f + f_3) \right] \]

where:
- The delta function \( \delta(f) \) represents the constant term.
- The cosine term \( 0.5 \cos(2 \pi f_{1} t) \) contributes to delta functions at \( \pm f_1 \).
- The sine terms contribute to delta functions at \( \pm f_2 \) and \( \pm f_3 \), with imaginary coefficients indicating phase shifts.

\subsection*{Python Implementation and Plot}
The plot Figure~\ref{fig:ex1_a_plot} below illustrates the time-domain signal, and Figure~\ref{fig:ex1_a_fft} shows its Fourier transform:

\begin{figure}[h]
    \centering
    \includegraphics[width=0.8\textwidth]{fig/ex1_a_plot.png}
    \caption{Analog Signal $x(t)$}
    \label{fig:ex1_a_plot}
\end{figure}

\begin{figure}[h]
    \centering
    \includegraphics[width=0.8\textwidth]{fig/ex1_a_fft_stem.png}
    \caption{Fourier Transform of $x(t)$}
    \label{fig:ex1_a_fft}
\end{figure}

\subsection*{Conclusion}
The Fourier transform of the given signal shows spikes at the frequencies \( f_1 = 2 \, \text{kHz} \), \( f_2 = 4 \, \text{kHz} \), and \( f_3 = 6 \, \text{kHz} \), corresponding to the cosine and sine terms in the signal. The plot of the analog signal \( x(t) \) illustrates its time-domain behavior over the interval from 0 to 1 ms, and the Fourier transform plot shows the frequency-domain representation with spikes at the expected frequencies.

            %! Author = wolfram_e_laube
%! Date = 06.05.24

\item[(b)]
\section{Task (b): Fourier Transform and Spectrum Visualization}

\subsection{Fourier Transform of $x(t)$}
Given the signal $x(t) = \sin(2\pi 4000 t) + \sin(2\pi 6000 t)$, we calculate the Fourier transform, which reveals delta functions at the frequencies of the sinusoidal components:
$$
X(f) = \frac{1}{2i} \left(\delta(f - 4000) - \delta(f + 4000) + \delta(f - 6000) - \delta(f + 6000)\right)
$$
This expression indicates that the spectrum of $x(t)$ consists of spikes at $\pm 4000$ Hz and $\pm 6000$ Hz.

\subsection{Spectrum Visualization and Sampling Effects}
The spectrum is then visualized, taking into account the sampling frequency $f_s = 10,000$ Hz, which leads to periodic replication of the spectrum. The replication and the combination of these spectra due to sampling are illustrated to show how aliasing could affect the resultant digital signal $x[n]$. The effects are demonstrated using a Python script that plots the original and shifted spectra within the range $-f_s$ to $+f_s$.

\begin{figure}[h]
    \centering
    \includegraphics[width=0.49\textwidth]{fig/ex1_b_plot}
    \caption{Spectrum of \(x(t)\)}
    \label{fig:ex1_b_plot}
\end{figure}

The shifted spectrum plot will show the spectral shifts due to the sampling frequency and illustrate the overlapping spectra.
            %! Author = wolfram_e_laube
%! Date = 06.05.24

\item[(c)]
The Python code accomplishing this is:

\begin{verbatim}
import numpy as np
import matplotlib.pyplot as plt

# Parameters
fs_analog = 100e3  # 100 kHz
fs = 10e3  # 10 kHz
f1 = 4e3  # 4 kHz
f2 = 6e3  # 6 kHz
t_end = 2e-3  # 2 ms

# Time vectors
t = np.arange(0, t_end, 1/fs_analog)
n = np.arange(0, int(t_end * fs))

# Analog signal
x_t = np.sin(2 * np.pi * f1 * t) + np.sin(2 * np.pi * f2 * t)

# Sampled signal
x_n = np.sin(2 * np.pi * f1 * n / fs) + np.sin(2 * np.pi * f2 * n / fs)

# Plotting
plt.figure()
plt.plot(t * 1e3, x_t, label='Analog signal x(t)')
plt.stem(n * 1e3 / fs, x_n, linefmt='r', markerfmt='ro', basefmt=' ', label='Sampled signal x[n]', use_line_collection=True)
plt.xlabel('Time (ms)')
plt.ylabel('Amplitude')
plt.title('Analog and Sampled Signals')
plt.legend()
plt.grid(True)
plt.show()
\end{verbatim}

\begin{figure}[h]
    \centering
    \includegraphics[width=0.49\textwidth]{fig/ex1_c_plot}
    \caption{Analog and Sampled Signals of \(x(t)\)}
    \label{fig:ex1_c_plot}
\end{figure}

        \end{enumerate}

    \end{aufgabe}

    \begin{aufgabe}{DFT Theory (20\%)}

        100 values of an analog signal $x(t)$ were measured with a sampling time of $1 \mathrm{~ms}$, leading to the discrete-time signal $x[n]$.
        This time domain signal $x[n]$ is transformed to frequency domain using the DFT/FFT, i.e., a 100-point DFT/FFT is calculated.

        \begin{enumerate}
            \item[(a)] What is the frequency spacing between two neighboring spectral points in the DFT spectrum, i.e., what is the frequency resolution?

            \item[(b)] What is the "period" of the DFT spectrum in terms of samples, in terms of frequency, and in terms of normalized angular frequency?

            We now append zeros to the discrete-time signal $x[n]$ to obtain a signal that has a length of 128 samples.

            \item[(c)] Why do we append exactly so many zeros, so that a total signal of length $128=2^{7}$ results?

            \item[(d)] What is the frequency spacing between two neighboring spectral points in the DFT spectrum now?

            \item[(e)] How can the changed distance between two neighboring spectral points be interpreted?
        \end{enumerate}

        \hrule

        \begin{enumerate}
            \item[(a)]
\section{Block Diagram of the LTI System}

\subsection*{Problem Statement}
Sketch the block diagram of the LTI system corresponding to the given difference equation:
\[ y[n] = x[n] - \frac{1}{15} y[n-1] + \frac{2}{5} y[n-2] \]

\subsection*{Theoretical Background}
A block diagram for an LTI (Linear Time-Invariant) system visually represents the flow of signals and the operations applied to them. The difference equation can be transformed into a block diagram using delay elements \( z^{-1} \), multipliers, and adders.

\subsection*{Mathematical Derivation}
The given difference equation is:
\[ y[n] = x[n] - \frac{1}{15} y[n-1] + \frac{2}{5} y[n-2] \]

This can be expressed in terms of block diagram components:
\begin{itemize}
    \item \( x[n] \) is the input signal.
    \item \( y[n] \) is the output signal.
    \item \( y[n-1] \) and \( y[n-2] \) are delayed versions of the output signal.
    \item Multipliers \( -\frac{1}{15} \) and \( \frac{2}{5} \) scale the delayed outputs.
    \item An adder combines the scaled delayed outputs with the input signal to produce the output.
\end{itemize}

\subsection*{Block Diagram}
The block diagram representation is as follows:

\begin{figure}[h]
    \centering
    \includegraphics[width=0.8\textwidth]{fig/ex2_a_block_diagram.png}
    \caption{Block Diagram of the LTI System}
    \label{fig:ex2_a_block_diagram}
\end{figure}

\subsection*{Conclusion}
The block diagram illustrates the flow of the input signal \( x[n] \), the delayed outputs \( y[n-1] \) and \( y[n-2] \), and their scaled versions being combined to form the output signal \( y[n] \).

            %! Author = wolfram_e_laube
%! Date = 06.05.24

\item[(b)]
\begin{figure}[h]
    \centering
    \includegraphics[width=0.49\textwidth]{fig/ex2_b_plot}
    \caption{Spectrum of \(x(t)\)}
    \label{fig:ex2_b_plot}
\end{figure}

            \item[(c)]
\section{Transfer Function}

\subsection*{Problem Statement}
Compute the transfer function \( H(z) = \frac{Y(z)}{X(z)} \).

\subsection*{Theoretical Background}
The transfer function \( H(z) \) of a system characterizes its input-output relationship in the z-domain. It is obtained by taking the z-transform of the difference equation and solving for \( H(z) = \frac{Y(z)}{X(z)} \).

\subsection*{Mathematical Derivation}
Given the difference equation:
\[ y[n] = x[n] - \frac{1}{15} y[n-1] + \frac{2}{5} y[n-2] \]

Taking the z-transform of both sides:
\[ Y(z) = X(z) - \frac{1}{15} Y(z) z^{-1} + \frac{2}{5} Y(z) z^{-2} \]

Rearranging to solve for \( \frac{Y(z)}{X(z)} \):
\[ Y(z) + \frac{1}{15} Y(z) z^{-1} - \frac{2}{5} Y(z) z^{-2} = X(z) \]
\[ Y(z) \left( 1 + \frac{1}{15} z^{-1} - \frac{2}{5} z^{-2} \right) = X(z) \]
\[ H(z) = \frac{Y(z)}{X(z)} = \frac{1}{1 + \frac{1}{15} z^{-1} - \frac{2}{5} z^{-2}} \]

Multiplying the numerator and denominator by \( z^2 \) to get a standard form:
\[ H(z) = \frac{z^2}{z^2 + \frac{1}{15} z - \frac{2}{5}} \]

\subsection*{Conclusion}
The transfer function \( H(z) \) of the system is:
\[ H(z) = \frac{z^2}{z^2 + \frac{1}{15} z - \frac{2}{5}} \]
This transfer function characterizes the input-output relationship of the given difference equation in the z-domain.

            \item[(d)]
\section*{Task (d)}

\subsection*{Problem Statement}
What is the frequency spacing between two neighboring spectral points in the DFT spectrum now, after zero-padding to a length of 128 samples?

\subsection*{Theoretical Background}
The frequency resolution of the Discrete Fourier Transform (DFT) is determined by the sampling frequency \( f_s \) and the number of points \( N \) in the DFT. The frequency resolution \( \Delta f \) is given by:

\[ \Delta f = \frac{f_s}{N} \]

Zero-padding a signal to increase its length \( N \) affects the frequency resolution of the DFT. With an increased length, the frequency resolution improves, meaning the spacing between neighboring spectral points becomes finer.

\subsection*{Mathematical Derivation}
Given:
\begin{itemize}
    \item The sampling time \( T_s = 1 \, \text{ms} \)
    \item The new number of points \( N = 128 \)
\end{itemize}

First, calculate the sampling frequency \( f_s \):

\[ f_s = \frac{1}{T_s} = \frac{1}{1 \times 10^{-3} \, \text{s}} = 1000 \, \text{Hz} \]

Next, calculate the new frequency resolution \( \Delta f \):

\[ \Delta f = \frac{f_s}{N} = \frac{1000 \, \text{Hz}}{128} \approx 7.8125 \, \text{Hz} \]

Thus, the frequency spacing between two neighboring spectral points in the DFT spectrum after zero-padding to a length of 128 samples is approximately \( 7.8125 \, \text{Hz} \).

\subsection*{Conclusion}
The frequency resolution of the 128-point DFT/FFT calculated from the zero-padded discrete-time signal \( x[n] \) is approximately \( 7.8125 \, \text{Hz} \). This means that each spectral point in the DFT spectrum is now spaced by \( 7.8125 \, \text{Hz} \), providing a finer frequency resolution compared to the original 100-point DFT.

            \item[(e)]
\section{System Stability}

\subsection*{Problem Statement}
Is the system stable? Explain your answer.

\subsection*{Theoretical Background}
A discrete-time system is stable if its poles lie inside the unit circle in the z-plane. This means that for a system described by its transfer function \( H(z) \), the system is stable if all the poles of \( H(z) \) satisfy \( |z| < 1 \).

\subsection*{Mathematical Derivation}
From Task (d), we have the poles of the transfer function \( H(z) \):
\[ z = \frac{3}{5}, \quad z = -\frac{2}{3} \]

To determine the stability of the system, we need to check if the magnitudes of these poles are less than 1.

\begin{itemize}
    \item For \( z = \frac{3}{5} \):
    \[ \left| \frac{3}{5} \right| = 0.6 < 1 \]
    \item For \( z = -\frac{2}{3} \):
    \[ \left| -\frac{2}{3} \right| = 0.67 < 1 \]
\end{itemize}

Since both poles have magnitudes less than 1, they lie inside the unit circle.

\subsection*{Conclusion}
The system is stable because all the poles of the transfer function \( H(z) \) lie inside the unit circle in the z-plane.

        \end{enumerate}

    \end{aufgabe}

    \begin{aufgabe}{FFT in Image Processing (15\%)}

        In image processing, besides other methods, also the FFT can be utilized to conduct edge detection.
        To this end, a two-dimensional FFT (MATLAB command: $fft2$ ) is calculated from an image to transform it from spatial domain to frequency domain. Here, edges in an image produce high frequencies in the corresponding spectrum.

        In this exercise, you should make the edges of an image visible.
        Start by making yourself familiar with the MatLAB script fft\_edge\_detection.
        Now filter the image in frequency domain, such that the edges become visible after a transformation back to spatial domain.
        Describe your approach and justify why it works.
        A solution could look like the one depicted in Fig. \ref{fig:ex2_fft}.

        \begin{figure}[h]
            \centering
            \includegraphics[width=0.90\textwidth]{fig/ex2_fft}
            \caption{Example for edge detection with the FFT}
            \label{fig:ex2_fft}
        \end{figure}

        \hrule

%        \begin{enumerate}
            \section*{Exercise 3: Edge Detection Using FFT}

\subsection*{Problem Statement}
In image processing, besides other methods, the FFT can be utilized to conduct edge detection. To this end, a two-dimensional FFT (MATLAB command: `fft2`) is calculated from an image to transform it from the spatial domain to the frequency domain. Here, edges in an image produce high frequencies in the corresponding spectrum.

In this exercise, you should make the edges of an image visible. Start by making yourself familiar with the MATLAB script `fft_edge_detection`. Now filter the image in the frequency domain, such that the edges become visible after a transformation back to the spatial domain. Describe your approach and justify why it works. A solution could look like the one depicted in Fig. 1.

\subsection*{Python Script}
\begin{verbatim}
import numpy as np
import matplotlib.pyplot as plt
from scipy.fft import fft2, ifft2, fftshift
from imageio import imread

# Read the image
image_path = 'image.png'
image = imread(image_path, mode='L')

# Perform the 2D FFT
image_fft = fft2(image)
image_fft_shifted = fftshift(image_fft)  # Shift the zero frequency component to the center

# Create a high-pass filter
rows, cols = image.shape
crow, ccol = rows // 2 , cols // 2  # Center of the image

# Create a mask with high value at edges
high_pass_mask = np.ones((rows, cols), dtype=np.float32)
high_pass_radius = 30  # Adjust the radius to control the filter size
center_radius = high_pass_radius

for i in range(rows):
    for j in range(cols):
        if np.sqrt((i - crow)**2 + (j - ccol)**2) < center_radius:
            high_pass_mask[i, j] = 0

# Apply the high-pass filter
filtered_fft = image_fft_shifted * high_pass_mask

# Inverse FFT to transform back to the spatial domain
filtered_fft_shifted_back = fftshift(filtered_fft)
image_filtered = np.abs(ifft2(filtered_fft_shifted_back))

# Plot the original and filtered images
plt.figure(figsize=(12, 6))

plt.subplot(1, 2, 1)
plt.imshow(image, cmap='gray')
plt.title('Original Image')
plt.axis('off')

plt.subplot(1, 2, 2)
plt.imshow(image_filtered, cmap='gray')
plt.title('Edge Detection using FFT')
plt.axis('off')

# Save the plot for LaTeX inclusion
plt.savefig('fig/ex3_edge_detection_fft.png')
plt.show()
\end{verbatim}

\subsection*{Edge Detection using FFT}
\begin{figure}[h]
    \centering
    \includegraphics[width=0.8\textwidth]{fig/ex3_edge_detection_fft.png}
    \caption{Edge Detection using FFT}
    \label{fig:ex3_edge_detection_fft}
\end{figure}

\subsection*{Approach and Justification}
The approach involves the following steps:
\begin{enumerate}
    \item Perform a 2D FFT on the image to transform it from the spatial domain to the frequency domain.
    \item Apply a high-pass filter in the frequency domain to retain high-frequency components (which correspond to edges) and suppress low-frequency components (which correspond to smooth regions).
    \item Perform an inverse FFT to transform the filtered image back to the spatial domain.

This approach works because edges in an image correspond to abrupt changes in intensity, which translate to high frequencies in the frequency domain. By applying a high-pass filter, we can isolate these high-frequency components, making the edges visible in the spatial domain after the inverse FFT.

%        \end{enumerate}

    \end{aufgabe}

    \begin{aufgabe}{FFT in Audio Signal Processing - Short Time Fourier Transform (30\%)}
        In the Dual Tone Multiple Frequency (DTMF) method, each of the 16 possible information symbols $q \in\{0,1,2,3,4,5,6,7,8,9, A, B, C, D, *, \#\}$ is represented by a superposition of two sinusoidal audio signals with different frequencies.
        The assignment of every symbols to the two frequencies contained in the corresponding audio signal is given in Tab. \ref{tab:ex4_frequencies}.
        The duration of a single symbol is $70 \mathrm{~ms}$, and the audio signals of two symbols are separated by a pause of $30 \mathrm{~ms}$.
        The file $\mathrm{dtmf.wav}$ contains a signal consisting of a sequence DTMF signals corresponding to a sequence of randomly chosen symbols, where the signal has been generated with a sampling frequency of $f_{s}=8 \mathrm{kHz}$.

        \begin{table}
            \caption{DTMF frequencies}
            \label{tab:ex4_frequencies}
            \centering
            \begin{tabular}{|c|c|c|c|c|}
                \hline
                & $1209 \mathrm{~Hz}$ & $1336 \mathrm{~Hz}$ & $1477 \mathrm{~Hz}$ & $1633 \mathrm{~Hz}$ \\
                \hline
                $697 \mathrm{~Hz}$ & 1                   & 2                   & 3                   & $A$                 \\
                \hline
                $770 \mathrm{~Hz}$ & 4                   & 5                   & 6                   & $B$                 \\
                \hline
                $852 \mathrm{~Hz}$ & 7                   & 8                   & 9                   & $C$                 \\
                \hline
                $941 \mathrm{~Hz}$ & $*$                 & 0                   & $\#$                & $D$                 \\
                \hline
            \end{tabular}
        \end{table}


        \begin{enumerate}
            \item[(a)] Write a MatLAB script, where you read in the signal and the sampling frequency from the file dtmf.wav. Hint: the MatLaB function audioread might be useful. Plot a $300 \mathrm{~ms}$ long segment of the signal and put the resulting plot into your protocol (including a correct labeling of the axes, etc.). Given this signal segment in time domain, can you make any statement which symbols are contained in this segment?
            \item[(b)] Compute the spectrum for the whole signal length using the MATLAB command fft.
            Plot the magnitude of the spectrum.
            Label the axes correctly, with the frequency axis scale in $\mathrm{Hz}$ (Hint: the frequency values given in Tab. 1 should be visible at the correct position on the frequency axis).

            \item[(c)] Now the so-called short-time Fourier Transform (STFT) should be implemented.
            To this end, the total signal is grouped into overlapping blocks of a specified length, and each block is transformed individually into frequency domain.
            Additionally, the individual blocks should be filtered using a Hamming window (MATLAB command hamming) for improving the spectral illustration.
            \begin{enumerate}
                \item[-] Implement the generation of the signal blocks given the total signal.
                Every block should have a length of 256 samples.
                Use an overlapping factor of 2, i.e., every block is overlapping with its preceding block by half the signal length.
                The last block should also have a length of 256 , i.e., if not enough samples from the total signal are left to obtain a block of length 256 , append the appropriate number of zeros.
                \item[-] Every block should be multiplied with a Hamming window.
                The window can be generated with the MATLAB command hamming.
                \item[-] Transform the windowed blocks to frequency domain using the FFT to obtain the Fourier transformed blocks (FTBs).
                \item[-] Plot a 2d diagram showing the FTBs. For this purpose, consider which frequency values have to be displayed on the frequency axis, and which time values have to be plotted on the time axis, i.e., which frequency range has been regarded with the FFTs and at which points in time have the FFTs of the individual blocks been computed?
                Generate two MATLAB vectors $f_{stft}$ and $t_{stft}$ , which contain the appropriate frequency and time values, respectively.
                Then, plot the diagram containing the magnitude of the STFT result using the MATLAB command surface ( $t_{stft}$, $f_{stft}$, $abs(stft\_mtx)$), where $stft\_mtx$ is the matrix containing the individual FTBs. Display the amplitude information with the command \texttt{colorbar}.
                The STFT magnitude diagram in your protocol should look similar as in Fig \ref{fig:ex4_c_stft_magnitude} (the symbol sequence might not be the same), where in Fig \ref{fig:ex4_c_stft_magnitude} only the frequency range $0 - 2000 \mathrm{~Hz}$ is shown.

                \begin{figure}[h]
                    \centering
                    \includegraphics[width=0.49\textwidth]{fig/ex4_c_stft_magnitude}
                    \caption{STFT magnitude of a DTMF sequence}
                    \label{fig:ex4_c_stft_magnitude}
                \end{figure}
            \end{enumerate}
            \item[(d)] Perform the same steps as in (c), but without multiplying the signal blocks by a Hamming window.
            How does the resulting magnitude diagram of the STFT differ to the one computed in (c)?
            How is the effect called that causes this difference?

            \item[(e)] On basis of the plotted diagram in (c), determine the symbol sequence that has been used for generating the total signal.

            \item[(f)] Answer the following questions:
            \begin{enumerate}
                \item[-] What is the essential difference between the diagrams plotted in (b) and (c), and what becomes apparent in the diagram in (c) that cannot be observed from the diagram in (b)?
                \item[-] Give an example for an application of the STFT and describe it briefly.
            \end{enumerate}
        \end{enumerate}

        \hrule

\begin{enumerate}
        %! Author = wolfram_e_laube
%! Date = 16.04.24

\item[(a)]
A Python function to calculate the DTFT of a finite sequence is provided below.
The function employs NumPy's vectorized operations to avoid explicit for-loops.

\begin{verbatim}
import numpy as np

def dtft(x, n, w):
    """
    Compute the Discrete-time Fourier Transform (DTFT) of a finite sequence.

    :param x: Finite duration sequence over n (numpy array)
    :param n: Sample position vector (numpy array)
    :param w: Frequency location vector (numpy array)
    :return: DTFT values computed at w frequencies (numpy array)
    """
    # Convert all inputs to numpy arrays to ensure proper calculations
    x = np.array(x)
    n = np.array(n)
    w = np.array(w)

    # Create a 2D meshgrid for the frequencies and samples for broadcasting
    N, W = np.meshgrid(n, w)

    # Compute the DTFT using broadcasting and vectorized operations
    X = np.exp(-1j * N * W) @ x
    return X
\end{verbatim}

This function calculates the DTFT using matrix multiplication, which is a vectorized operation
that can replace explicit looping constructs.

        \item[(b)]
\section*{Task (b)}

\subsection*{Problem Statement}
Compute the spectrum for the whole signal length using the MATLAB command `fft`. Plot the magnitude of the spectrum. Label the axes correctly, with the frequency axis scale in Hz (Hint: the frequency values given in Table 1 should be visible at the correct position on the frequency axis).

\subsection*{Python Script}
\begin{verbatim}
import numpy as np
import matplotlib.pyplot as plt
from scipy.io import wavfile
import os

# Create fig directory if it doesn't exist
if not os.path.exists('fig'):
    os.makedirs('fig')

# Read the DTMF signal from the WAV file
fs, signal = wavfile.read('dtmf.wav')

# Compute the FFT for the whole signal length
spectrum = np.fft.fft(signal)
frequencies = np.fft.fftfreq(len(signal), 1/fs)

# Plot the magnitude spectrum
plt.figure(figsize=(10, 6))
plt.plot(frequencies[:len(frequencies) // 2], np.abs(spectrum[:len(frequencies) // 2]))
plt.title('Magnitude Spectrum of the Entire DTMF Signal')
plt.xlabel('Frequency (Hz)')
plt.ylabel('Magnitude')
plt.grid(True)
plt.savefig('fig/ex4_b_dtmf_spectrum.png')
plt.show()
\end{verbatim}

\subsection*{Magnitude Spectrum of the Entire DTMF Signal}
\begin{figure}[h]
    \centering
    \includegraphics[width=0.8\textwidth]{fig/ex4_b_dtmf_spectrum.png}
    \caption{Magnitude Spectrum of the Entire DTMF Signal}
    \label{fig:ex4_b_dtmf_spectrum}
\end{figure}

\subsection*{Analysis}
The magnitude spectrum of the entire DTMF signal shows the frequencies present in the signal. The frequencies corresponding to the DTMF tones listed in Table 1 should be visible at the correct positions on the frequency axis, confirming the presence of these tones in the signal.

        %! Author = wolfram_e_laube
%! Date = 16.04.24

\item[(c)]
To observe the impact of varying $\Omega$, the Python code below alters the frequency resolution and compares the results:

\begin{verbatim}
import matplotlib.pyplot as plt
import numpy as np

# Varying the frequency resolution
w_coarse = np.linspace(-np.pi, np.pi, 100)
w_fine = np.linspace(-np.pi, np.pi, 1600)
X_coarse = dtft(x, n, w_coarse)
X_fine = dtft(x, n, w_fine)

# Plot the magnitude response for both resolutions
plt.figure(figsize=(14, 5))
plt.subplot(1, 2, 1)
plt.plot(w_coarse, np.abs(X_coarse))
plt.title('Coarse Frequency Resolution')
plt.xlabel('Frequency (rad/sample)')
plt.ylabel('Magnitude')
plt.grid()

plt.subplot(1, 2, 2)
plt.plot(w_fine, np.abs(X_fine))
plt.title('Fine Frequency Resolution')
plt.xlabel('Frequency (rad/sample)')
plt.ylabel('Magnitude')
plt.grid()

plt.tight_layout()
plt.show()
\end{verbatim}

The DTFT results are calculated and plotted with both coarse and fine frequency resolutions to observe differences in the spectrum.
A finer resolution unveils more details in the frequency domain representation of the signal.

        \item[(d)]
\section*{Task (d)}

\subsection*{Problem Statement}
Perform the same steps as in (c), but without multiplying the signal blocks by a Hamming window. How does the resulting magnitude diagram of the STFT differ from the one computed in (c)? How is the effect called that causes this difference?

\subsection*{Python Script}
\begin{verbatim}
import numpy as np
import matplotlib.pyplot as plt
from scipy.io import wavfile
import os

# Create fig directory if it doesn't exist
if not os.path.exists('fig'):
    os.makedirs('fig')

# Read the DTMF signal from the WAV file
fs, signal = wavfile.read('dtmf.wav')

# Parameters for STFT
block_length = 256  # Block length
overlap = block_length // 2  # Overlapping factor of 2

# Pad the signal with zeros if necessary
padding_length = (block_length - len(signal) % block_length) % block_length
padded_signal = np.append(signal, np.zeros(padding_length))

# Generate signal blocks
num_blocks = (len(padded_signal) - overlap) // (block_length - overlap)
blocks = np.zeros((num_blocks, block_length))

for i in range(num_blocks):
    start = i * (block_length - overlap)
    end = start + block_length
    blocks[i, :] = padded_signal[start:end]

# Compute the FFT for each block
ftbs = np.fft.fft(blocks, axis=1)

# Generate the time and frequency vectors for plotting
f_stft = np.fft.fftfreq(block_length, 1/fs)[:block_length // 2]
t_stft = np.arange(num_blocks) * (block_length - overlap) / fs

# Plot the STFT magnitude without Hamming window
plt.figure(figsize=(12, 6))
plt.pcolormesh(t_stft, f_stft, np.abs(ftbs[:, :block_length // 2].T), shading='gouraud')
plt.title('STFT Magnitude of DTMF Signal Without Hamming Window')
plt.xlabel('Time (s)')
plt.ylabel('Frequency (Hz)')
plt.colorbar(label='Magnitude')
plt.ylim(0, 2000)  # Limit the frequency range to 0-2000 Hz
plt.savefig('fig/ex4_d_stft_magnitude_no_hamming.png')
plt.show()
\end{verbatim}

\subsection*{STFT Magnitude of the DTMF Signal Without Hamming Window}
\begin{figure}[h]
    \centering
    \includegraphics[width=0.8\textwidth]{fig/ex4_d_stft_magnitude_no_hamming.png}
    \caption{STFT Magnitude of the DTMF Signal Without Hamming Window}
    \label{fig:ex4_d_stft_magnitude_no_hamming}
\end{figure}

\subsection*{Analysis}
The resulting magnitude diagram of the STFT without the Hamming window shows more spectral leakage compared to the one computed with the Hamming window in Task (c). Spectral leakage occurs because the signal blocks are treated as if they are periodic, leading to discontinuities at the block edges. These discontinuities cause the energy to spread across multiple frequencies, resulting in a less clear representation of the signal's frequency content.

        \item[(e)]
\section{FIR Filter Design with Rectangular Window}

\subsection*{Problem Statement}
Design an FIR filter of order \( N = 20 \) with a rectangular window with a corner radian frequency \( \Omega_0 \). Plot its frequency response and the tolerance scheme in one plot. To this end, complete the provided file `dsp_5_4.m`.

\subsection*{Theoretical Background}
The rectangular window is a simple window function that can be used to truncate the ideal impulse response to a finite length. The frequency response of the resulting FIR filter can then be analyzed and compared with the specified tolerance scheme.

\subsection*{Mathematical Derivation}
The ideal impulse response is given by:
\[ h_{\text{ideal}}[n] = 0.37 \, \text{sinc}(0.37n) \]

To design an FIR filter of order \( N = 20 \) with a rectangular window:
\begin{itemize}
    \item Generate the ideal impulse response \( h_{\text{ideal}}[n] \) for \( n = -10 \) to \( 10 \) (centered around zero).
    \item Apply the rectangular window, which in this case does not change the values of \( h_{\text{ideal}}[n] \) because the window is entirely non-zero over the range.
    \item The resulting windowed and shifted impulse response is the FIR filter coefficients.
\end{itemize}

\subsection*{Implementation and Results}
The frequency response of the designed FIR filter is computed and plotted using Python. The plot below illustrates the frequency response along with the specified tolerance scheme.

\begin{figure}[h]
    \centering
    \includegraphics[width=0.8\textwidth]{fig/ex4_e_frequency_response.png}
    \caption{Frequency Response of FIR Filter with Rectangular Window}
    \label{fig:ex4_e_frequency_response}
\end{figure}

\subsection*{Conclusion}
The frequency response plot shows the performance of the FIR filter designed with a rectangular window of order \( N = 20 \). The tolerance scheme is also illustrated for comparison.

        \item[(f)]
\section{Tolerance Scheme and Filter Order}

\subsection*{Problem Statement}
Is the tolerance scheme being violated? Can the tolerance scheme be fulfilled by increasing the filter order to \( N = 90 \)?

\subsection*{Theoretical Background}
The tolerance scheme specifies the allowable deviations in the passband and stopband of the filter's frequency response. By comparing the frequency response of the designed FIR filter with the specified tolerance scheme, we can determine if the scheme is violated. Increasing the filter order typically results in a sharper transition band and improved adherence to the tolerance scheme.

\subsection*{Analysis}

\subsubsection*{Current Filter Order \( N = 20 \)}
The frequency response of the FIR filter with \( N = 20 \) was previously plotted. By examining this plot, we can check for any violations of the tolerance scheme in the passband ripple and stopband attenuation.

\subsubsection*{Increased Filter Order \( N = 90 \)}
An FIR filter with \( N = 90 \) is designed using the same method (rectangular window). The frequency response is computed and plotted to determine if it meets the tolerance scheme requirements.

\subsection*{Implementation and Results}
The frequency response of the FIR filter with \( N = 90 \) is computed and plotted using Python. The plot below illustrates the frequency response along with the specified tolerance scheme.

\begin{figure}[h]
    \centering
    \includegraphics[width=0.8\textwidth]{fig/ex4_f_frequency_response_90.png}
    \caption{Frequency Response of FIR Filter with Rectangular Window (N=90)}
    \label{fig:ex4_f_frequency_response_90}
\end{figure}

\subsection*{Conclusion}
By comparing the frequency response plots for \( N = 20 \) and \( N = 90 \), we can determine if the tolerance scheme is violated for \( N = 20 \) and if increasing the filter order to \( N = 90 \) fulfills the tolerance scheme requirements.

\end{enumerate}

    \end{aufgabe}

    \begin{aufgabe}{Window effects of the DFT (20\%)}

        Let us consider a signal consisting of two cosine oscillations with close frequencies.
        The actually infinite signal is time limited by windowing it once with a rectangular window and once with a Hamming window of length N.
        Generate the signal in MATLAB as follows:

        \begin{lstlisting}[label={lst:ex5_lstlisting}]
N = 128;
n = 0:N-1;
w1 = 2*pi*0.1;
w2 = 2*pi*0.15;@
x = cos(w1*n) + cos(w2*n);
        \end{lstlisting}

        Since $x$ has only finite length, it implicitly has already been windowed with a rectangular window.
        \begin{enumerate}
            \item[(a)] Compute the (discrete) spectrum of $x$ and plot a line plot of its magnitude (MATLAB commands $f f t$, abs, stem). Do not forget to label the axes!

            \item[(b)] Generate a Hamming window of length $N$ using the MatLAB command hamming.
            Multiply the Hamming window with the signal $\mathrm{x}$ to obtain the signal $\mathrm{y}$.
            Compute the spectrum of $\mathrm{y}$ and display its magnitude like in (a).

            \item[(c)] Compare and interpret the results from (a) and (b).

            \item[(d)] Experiment with w1 and w2 (i.e., adjust their values) and find a setting, where the DFT/FFT yields the exact result.
            Explain why the DFT/FFT result is exact with the selected settings.
        \end{enumerate}
        \hrule

\begin{enumerate}
        \item[(a)]
\section*{Task (a)}

\subsection*{Problem Statement}
Let us consider a signal consisting of two cosine oscillations with close frequencies. The actually infinite signal is time limited by windowing it once with a rectangular window and once with a Hamming window of length \( N \). Generate the signal in MATLAB as follows:

\begin{verbatim}
N = 128;
n = 0:N-1;
w1 = 2*pi*0.1;
w2 = 2*pi*0.15;
x = cos(w1*n) + cos(w2*n);
\end{verbatim}

Since \( x \) has only finite length, it implicitly has already been windowed with a rectangular window.

(a) Compute the (discrete) spectrum of \( x \) and plot a line plot of its magnitude (MATLAB commands `fft`, `abs`, `stem`). Do not forget to label the axes!

\subsection*{Python Script}
\begin{verbatim}
import numpy as np
import matplotlib.pyplot as plt

# Parameters
N = 128
n = np.arange(N)
w1 = 2 * np.pi * 0.1
w2 = 2 * np.pi * 0.15

# Generate the signal
x = np.cos(w1 * n) + np.cos(w2 * n)

# Compute the FFT
X = np.fft.fft(x)

# Compute the magnitude of the spectrum
magnitude_spectrum = np.abs(X)

# Generate frequency axis
frequencies = np.fft.fftfreq(N, d=1.0)

# Plot the magnitude spectrum
plt.figure(figsize=(10, 6))
plt.stem(frequencies, magnitude_spectrum, 'b', markerfmt=" ", basefmt="-b")
plt.title('Magnitude Spectrum of x')
plt.xlabel('Frequency (Hz)')
plt.ylabel('Magnitude')
plt.grid(True)
plt.show()

# Save the plot for LaTeX inclusion
plt.savefig('fig/ex5_a_magnitude_spectrum.png')
\end{verbatim}

\subsection*{Magnitude Spectrum of the Signal}
\begin{figure}[h]
    \centering
    \includegraphics[width=0.8\textwidth]{fig/ex5_a_magnitude_spectrum.png}
    \caption{Magnitude Spectrum of the Signal}
    \label{fig:ex5_a_magnitude_spectrum}
\end{figure}

\subsection*{Analysis}
The magnitude spectrum plot shows the discrete frequencies present in the signal. The peaks correspond to the frequencies of the two cosine components that make up the signal. The rectangular window implicitly applied by the finite length of the signal affects the resolution and clarity of the frequency peaks.

        \item[(b)]
\section*{Task (b)}

\subsection*{Problem Statement}
Generate a Hamming window of length \( N \) using the MATLAB command `hamming`. Multiply the Hamming window with the signal \( x \) to obtain the signal \( y \). Compute the spectrum of \( y \) and display its magnitude like in (a).

\subsection*{Python Script}
\begin{verbatim}
import numpy as np
import matplotlib.pyplot as plt

# Parameters
N = 128
n = np.arange(N)
w1 = 2 * np.pi * 0.1
w2 = 2 * np.pi * 0.15

# Generate the signal
x = np.cos(w1 * n) + np.cos(w2 * n)

# Generate the Hamming window
hamming_window = np.hamming(N)

# Multiply the Hamming window with the signal x to obtain y
y = x * hamming_window

# Compute the FFT of y
Y = np.fft.fft(y)

# Compute the magnitude of the spectrum
magnitude_spectrum_y = np.abs(Y)

# Generate frequency axis
frequencies = np.fft.fftfreq(N, d=1.0)

# Plot the magnitude spectrum of y
plt.figure(figsize=(10, 6))
plt.stem(frequencies, magnitude_spectrum_y, 'b', markerfmt=" ", basefmt="-b")
plt.title('Magnitude Spectrum of y (Hamming Window Applied)')
plt.xlabel('Frequency (Hz)')
plt.ylabel('Magnitude')
plt.grid(True)
plt.show()

# Save the plot for LaTeX inclusion
plt.savefig('fig/ex5_b_magnitude_spectrum_hamming.png')
\end{verbatim}

\subsection*{Magnitude Spectrum of the Signal with Hamming Window}
\begin{figure}[h]
    \centering
    \includegraphics[width=0.8\textwidth]{fig/ex5_b_magnitude_spectrum_hamming.png}
    \caption{Magnitude Spectrum of the Signal with Hamming Window Applied}
    \label{fig:ex5_b_magnitude_spectrum_hamming}
\end{figure}

\subsection*{Analysis}
The magnitude spectrum plot shows the discrete frequencies present in the signal after applying the Hamming window. The Hamming window reduces spectral leakage, resulting in a clearer representation of the frequency components compared to the rectangular window.

        \item[(c)]
\section*{Task (c)}

\subsection*{Problem Statement}
Compare and interpret the results from (a) and (b).

\subsection*{Comparison and Interpretation}

\subsubsection*{Comparison of Results}

\begin{itemize}
    \item \textbf{Result from Task (a)}: The magnitude spectrum of the signal \( x \) with a rectangular window.
    \item \textbf{Result from Task (b)}: The magnitude spectrum of the signal \( y \) with a Hamming window applied.
\end{itemize}

\subsubsection*{Interpretation}

\textbf{Rectangular Window (Task (a))}:
\begin{itemize}
    \item The rectangular window is essentially the default window applied due to the finite length of the signal \( x \).
    \item The magnitude spectrum shows clear peaks at the frequencies of the two cosine components. However, the rectangular window causes significant spectral leakage, where energy from the main frequencies spreads into neighboring frequencies. This results in wider and less distinct peaks, making it harder to precisely identify the exact frequencies of the components.
\end{itemize}

\textbf{Hamming Window (Task (b))}:
\begin{itemize}
    \item The Hamming window is designed to reduce spectral leakage by tapering the ends of the signal to zero. This reduces the discontinuities at the boundaries of the windowed signal.
    \item The magnitude spectrum with the Hamming window applied shows much sharper and more distinct peaks at the frequencies of the two cosine components. The reduction in spectral leakage results in narrower peaks, making it easier to identify the exact frequencies of the components.
    \item Although the peaks are sharper, the overall amplitude of the peaks may be slightly reduced compared to the rectangular window. This is a trade-off for achieving better frequency resolution and reducing spectral leakage.
\end{itemize}

        \item[(d)]
\section*{Task (d)}

\subsection*{Problem Statement}
Experiment with \( w1 \) and \( w2 \) (i.e., adjust their values) and find a setting where the DFT/FFT yields the exact result. Explain why the DFT/FFT result is exact with the selected settings.

\subsection*{Python Script}
\begin{verbatim}
import numpy as np
import matplotlib.pyplot as plt

# Parameters
N = 128
n = np.arange(N)

# Experiment with w1 and w2
w1 = 2 * np.pi * 0.125  # Corresponds to 1/8 of the sampling frequency
w2 = 2 * np.pi * 0.25   # Corresponds to 1/4 of the sampling frequency

# Generate the signal
x = np.cos(w1 * n) + np.cos(w2 * n)

# Compute the FFT
X = np.fft.fft(x)

# Compute the magnitude of the spectrum
magnitude_spectrum = np.abs(X)

# Generate frequency axis
frequencies = np.fft.fftfreq(N, d=1.0)

# Plot the magnitude spectrum
plt.figure(figsize=(10, 6))
plt.stem(frequencies, magnitude_spectrum, 'b', markerfmt=" ", basefmt="-b")
plt.title('Magnitude Spectrum of x with Exact DFT Frequencies')
plt.xlabel('Frequency (Hz)')
plt.ylabel('Magnitude')
plt.grid(True)

# Save the plot for LaTeX inclusion
plt.savefig('fig/ex5_d_magnitude_spectrum_exact.png')
plt.show()
\end{verbatim}

\subsection*{Magnitude Spectrum with Exact DFT Frequencies}
\begin{figure}[h]
    \centering
    \includegraphics[width=0.8\textwidth]{fig/ex5_d_magnitude_spectrum_exact.png}
    \caption{Magnitude Spectrum of the Signal with Exact DFT Frequencies}
    \label{fig:ex5_d_magnitude_spectrum_exact}
\end{figure}

\subsection*{Explanation}
By choosing \( w1 = 2 \pi \cdot 0.125 \) and \( w2 = 2 \pi \cdot 0.25 \), we align the signal frequencies with the DFT bin frequencies. This alignment ensures that each cosine component fits perfectly within the signal window, resulting in no spectral leakage and sharp peaks in the magnitude spectrum at the exact frequencies. The DFT/FFT yields exact results because the chosen frequencies are integer multiples of the fundamental frequency, allowing the DFT to capture the signal components without any distortion.

\end{enumerate}

    \end{aufgabe}

\end{document}
