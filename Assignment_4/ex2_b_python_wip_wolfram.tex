\item[(b)]
\section*{Task (b)}

\subsection*{Problem Statement}
What is the "period" of the DFT spectrum in terms of samples, in terms of frequency, and in terms of normalized angular frequency?

\subsection*{Theoretical Background}
The Discrete Fourier Transform (DFT) of a signal is periodic. This periodicity arises from the fact that the DFT assumes the input signal is one period of an infinitely periodic sequence. As a result, the frequency spectrum repeats itself.

1. **Period in Terms of Samples:**
   The DFT of a signal with \( N \) points is periodic with a period of \( N \) samples.

2. **Period in Terms of Frequency:**
   The frequency axis for the DFT is given by:
   \[
   f_k = \frac{k f_s}{N} \quad \text{for} \quad k = 0, 1, \ldots, N-1
   \]
   The period in terms of frequency is \( f_s \), the sampling frequency.

3. **Period in Terms of Normalized Angular Frequency:**
   The normalized angular frequency \( \omega \) is defined as:
   \[
   \omega = \frac{2\pi k}{N} \quad \text{for} \quad k = 0, 1, \ldots, N-1
   \]
   The period in terms of normalized angular frequency is \( 2\pi \).

\subsection*{Mathematical Derivation}
Given:
\begin{itemize}
    \item The number of points \( N = 100 \)
    \item The sampling frequency \( f_s = 1000 \, \text{Hz} \)
\end{itemize}

1. **Period in Terms of Samples:**
   The period of the DFT spectrum in terms of samples is \( N \):
   \[
   N = 100 \, \text{samples}
   \]

2. **Period in Terms of Frequency:**
   The period of the DFT spectrum in terms of frequency is \( f_s \):
   \[
   f_s = 1000 \, \text{Hz}
   \]

3. **Period in Terms of Normalized Angular Frequency:**
   The period of the DFT spectrum in terms of normalized angular frequency is \( 2\pi \):
   \[
   2\pi \, \text{radians}
   \]

\subsection*{Conclusion}
The period of the DFT spectrum is:
\begin{itemize}
    \item 100 samples in terms of the number of samples.
    \item 1000 Hz in terms of frequency.
    \item \( 2\pi \) radians in terms of normalized angular frequency.
\end{itemize}
