\item[(e)]
\section*{Task (e)}

\subsection*{Problem Statement}
How can the changed distance between two neighboring spectral points be interpreted?

\subsection*{Theoretical Background}
The frequency resolution of the Discrete Fourier Transform (DFT) is determined by the number of points \( N \) and the sampling frequency \( f_s \). The frequency resolution \( \Delta f \) is given by:

\[ \Delta f = \frac{f_s}{N} \]

Increasing the number of points \( N \) through zero-padding reduces the frequency spacing \( \Delta f \), providing a finer frequency resolution.

\subsection*{Mathematical Derivation}
Given:
\begin{itemize}
    \item The original number of points \( N = 100 \) with a frequency resolution of \( 10 \, \text{Hz} \)
    \item The zero-padded number of points \( N = 128 \) with a frequency resolution of \( 7.8125 \, \text{Hz} \)
\end{itemize}

The change in frequency resolution can be interpreted as follows:

\subsection*{Interpretation}
\begin{enumerate}
    \item **Improved Frequency Resolution:**
    The finer frequency resolution (\( 7.8125 \, \text{Hz} \) compared to \( 10 \, \text{Hz} \)) means that the DFT can now distinguish between frequencies that are closer together. This allows for a more detailed analysis of the signal's frequency components.

    \item **Better Spectral Detail:**
    With a finer frequency resolution, the DFT provides more spectral points within the same frequency range. This results in a more detailed and smoother representation of the signal's frequency spectrum.

    \item **Interpolation of Spectral Points:**
    Zero-padding effectively interpolates the existing spectral points, providing a clearer view of the underlying spectral shape. This is particularly useful for identifying narrowband components and subtle frequency variations.

    \item **Limitations:**
    It is important to note that zero-padding does not add new information to the signal. The actual frequency content of the signal remains unchanged. The improved resolution is a result of interpolating the DFT output, not capturing additional frequency components.
\end{enumerate}

\subsection*{Conclusion}
The changed distance between two neighboring spectral points, resulting from zero-padding, can be interpreted as an improved frequency resolution. This provides a more detailed and accurate representation of the signal's frequency spectrum, allowing for better analysis and interpretation of the signal's frequency components. However, it is crucial to understand that zero-padding does not introduce new information but enhances the clarity of the existing spectral information.
