\item[(d)]
\section*{Task (d)}

\subsection*{Problem Statement}
What is the frequency spacing between two neighboring spectral points in the DFT spectrum now, after zero-padding to a length of 128 samples?

\subsection*{Theoretical Background}
The frequency resolution of the Discrete Fourier Transform (DFT) is determined by the sampling frequency \( f_s \) and the number of points \( N \) in the DFT. The frequency resolution \( \Delta f \) is given by:

\[ \Delta f = \frac{f_s}{N} \]

Zero-padding a signal to increase its length \( N \) affects the frequency resolution of the DFT. With an increased length, the frequency resolution improves, meaning the spacing between neighboring spectral points becomes finer.

\subsection*{Mathematical Derivation}
Given:
\begin{itemize}
    \item The sampling time \( T_s = 1 \, \text{ms} \)
    \item The new number of points \( N = 128 \)
\end{itemize}

First, calculate the sampling frequency \( f_s \):

\[ f_s = \frac{1}{T_s} = \frac{1}{1 \times 10^{-3} \, \text{s}} = 1000 \, \text{Hz} \]

Next, calculate the new frequency resolution \( \Delta f \):

\[ \Delta f = \frac{f_s}{N} = \frac{1000 \, \text{Hz}}{128} \approx 7.8125 \, \text{Hz} \]

Thus, the frequency spacing between two neighboring spectral points in the DFT spectrum after zero-padding to a length of 128 samples is approximately \( 7.8125 \, \text{Hz} \).

\subsection*{Conclusion}
The frequency resolution of the 128-point DFT/FFT calculated from the zero-padded discrete-time signal \( x[n] \) is approximately \( 7.8125 \, \text{Hz} \). This means that each spectral point in the DFT spectrum is now spaced by \( 7.8125 \, \text{Hz} \), providing a finer frequency resolution compared to the original 100-point DFT.
