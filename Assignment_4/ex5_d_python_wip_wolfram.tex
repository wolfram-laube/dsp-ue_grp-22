\item[(d)]
\section*{Task (d)}

\subsection*{Problem Statement}
Experiment with \( w1 \) and \( w2 \) (i.e., adjust their values) and find a setting where the DFT/FFT yields the exact result. Explain why the DFT/FFT result is exact with the selected settings.

\subsection*{Python Script}
\begin{verbatim}
import numpy as np
import matplotlib.pyplot as plt

# Parameters
N = 128
n = np.arange(N)

# Experiment with w1 and w2
w1 = 2 * np.pi * 0.125  # Corresponds to 1/8 of the sampling frequency
w2 = 2 * np.pi * 0.25   # Corresponds to 1/4 of the sampling frequency

# Generate the signal
x = np.cos(w1 * n) + np.cos(w2 * n)

# Compute the FFT
X = np.fft.fft(x)

# Compute the magnitude of the spectrum
magnitude_spectrum = np.abs(X)

# Generate frequency axis
frequencies = np.fft.fftfreq(N, d=1.0)

# Plot the magnitude spectrum
plt.figure(figsize=(10, 6))
plt.stem(frequencies, magnitude_spectrum, 'b', markerfmt=" ", basefmt="-b")
plt.title('Magnitude Spectrum of x with Exact DFT Frequencies')
plt.xlabel('Frequency (Hz)')
plt.ylabel('Magnitude')
plt.grid(True)

# Save the plot for LaTeX inclusion
plt.savefig('fig/ex5_d_magnitude_spectrum_exact.png')
plt.show()
\end{verbatim}

\subsection*{Magnitude Spectrum with Exact DFT Frequencies}
\begin{figure}[h]
    \centering
    \includegraphics[width=0.8\textwidth]{fig/ex5_d_magnitude_spectrum_exact.png}
    \caption{Magnitude Spectrum of the Signal with Exact DFT Frequencies}
    \label{fig:ex5_d_magnitude_spectrum_exact}
\end{figure}

\subsection*{Explanation}
By choosing \( w1 = 2 \pi \cdot 0.125 \) and \( w2 = 2 \pi \cdot 0.25 \), we align the signal frequencies with the DFT bin frequencies. This alignment ensures that each cosine component fits perfectly within the signal window, resulting in no spectral leakage and sharp peaks in the magnitude spectrum at the exact frequencies. The DFT/FFT yields exact results because the chosen frequencies are integer multiples of the fundamental frequency, allowing the DFT to capture the signal components without any distortion.
