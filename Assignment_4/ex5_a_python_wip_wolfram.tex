\item[(a)]
\section*{Task (a)}

\subsection*{Problem Statement}
Let us consider a signal consisting of two cosine oscillations with close frequencies. The actually infinite signal is time limited by windowing it once with a rectangular window and once with a Hamming window of length \( N \). Generate the signal in MATLAB as follows:

\begin{verbatim}
N = 128;
n = 0:N-1;
w1 = 2*pi*0.1;
w2 = 2*pi*0.15;
x = cos(w1*n) + cos(w2*n);
\end{verbatim}

Since \( x \) has only finite length, it implicitly has already been windowed with a rectangular window.

(a) Compute the (discrete) spectrum of \( x \) and plot a line plot of its magnitude (MATLAB commands `fft`, `abs`, `stem`). Do not forget to label the axes!

\subsection*{Python Script}
\begin{verbatim}
import numpy as np
import matplotlib.pyplot as plt

# Parameters
N = 128
n = np.arange(N)
w1 = 2 * np.pi * 0.1
w2 = 2 * np.pi * 0.15

# Generate the signal
x = np.cos(w1 * n) + np.cos(w2 * n)

# Compute the FFT
X = np.fft.fft(x)

# Compute the magnitude of the spectrum
magnitude_spectrum = np.abs(X)

# Generate frequency axis
frequencies = np.fft.fftfreq(N, d=1.0)

# Plot the magnitude spectrum
plt.figure(figsize=(10, 6))
plt.stem(frequencies, magnitude_spectrum, 'b', markerfmt=" ", basefmt="-b")
plt.title('Magnitude Spectrum of x')
plt.xlabel('Frequency (Hz)')
plt.ylabel('Magnitude')
plt.grid(True)
plt.show()

# Save the plot for LaTeX inclusion
plt.savefig('fig/ex5_a_magnitude_spectrum.png')
\end{verbatim}

\subsection*{Magnitude Spectrum of the Signal}
\begin{figure}[h]
    \centering
    \includegraphics[width=0.8\textwidth]{fig/ex5_a_magnitude_spectrum.png}
    \caption{Magnitude Spectrum of the Signal}
    \label{fig:ex5_a_magnitude_spectrum}
\end{figure}

\subsection*{Analysis}
The magnitude spectrum plot shows the discrete frequencies present in the signal. The peaks correspond to the frequencies of the two cosine components that make up the signal. The rectangular window implicitly applied by the finite length of the signal affects the resolution and clarity of the frequency peaks.
