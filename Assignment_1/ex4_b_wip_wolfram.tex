%! Author = wolfram_e_laube
%! Date = 02.04.24

\item[b)]
System 2: $y(t)=x(t) \cdot \sin \left(\Omega_{0} t\right)$

\textbf{Linearity Test:}

For inputs $x_1(t)$ and $x_2(t)$, and constants $a$ and $b$, the system's response to the combination
$a x_1(t) + b x_2(t)$ is:

$$
y(t) = (a x_1(t) + b x_2(t)) \cdot \sin(\Omega_0 t)
$$

This simplifies to:

$$
y(t) = a x_1(t) \sin(\Omega_0 t) + b x_2(t) \sin(\Omega_0 t)
$$

This result is equal to $a y_1(t) + b y_2(t)$, indicating that the system satisfies the linearity property.

\textbf{Time Invariance Test:}

Consider an input $x(t - \tau)$ which produces the output:

$$
y(t) = x(t - \tau) \cdot \sin(\Omega_0 t)
$$

This output is not equivalent to $y(t - \tau) = x(t - \tau) \cdot \sin(\Omega_0 (t - \tau))$,
indicating that the system does not satisfy the time invariance property because the sine function's phase
is affected by the time shift.

\textbf{Conclusion for System 2:} The system is linear but not time-invariant.
