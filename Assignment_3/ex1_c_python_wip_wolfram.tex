%! Author = wolfram_e_laube
%! Date = 06.05.24

\item[(c)]
The Python code accomplishing this is:

\begin{verbatim}
import numpy as np
import matplotlib.pyplot as plt

# Parameters
fs_analog = 100e3  # 100 kHz
fs = 10e3  # 10 kHz
f1 = 4e3  # 4 kHz
f2 = 6e3  # 6 kHz
t_end = 2e-3  # 2 ms

# Time vectors
t = np.arange(0, t_end, 1/fs_analog)
n = np.arange(0, int(t_end * fs))

# Analog signal
x_t = np.sin(2 * np.pi * f1 * t) + np.sin(2 * np.pi * f2 * t)

# Sampled signal
x_n = np.sin(2 * np.pi * f1 * n / fs) + np.sin(2 * np.pi * f2 * n / fs)

# Plotting
plt.figure()
plt.plot(t * 1e3, x_t, label='Analog signal x(t)')
plt.stem(n * 1e3 / fs, x_n, linefmt='r', markerfmt='ro', basefmt=' ', label='Sampled signal x[n]', use_line_collection=True)
plt.xlabel('Time (ms)')
plt.ylabel('Amplitude')
plt.title('Analog and Sampled Signals')
plt.legend()
plt.grid(True)
plt.show()
\end{verbatim}

\begin{figure}[h]
    \centering
    \includegraphics[width=0.49\textwidth]{fig/ex1_c_plot}
    \caption{Analog and Sampled Signals of \(x(t)\)}
    \label{fig:ex1_c_plot}
\end{figure}
