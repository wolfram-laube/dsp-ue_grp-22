%! Author = wolfram_e_laube
%! Date = 06.05.24

\item[(b)]
\subsection{Task (b): Analysis of Signal Power with Zero-Order Hold Reconstruction}

\subsubsection{Problem Statement}
Determine the power in decibels (dB) of the baseband sinewave and the first out-of-band sinewave resulting from a zero-order hold (ZOH) DAC conversion of the discrete-time signal
$$
x[n] = \sqrt{2} \cdot \sin\left(2\pi \frac{1}{8}n\right)
$$
The DAC operates at a sampling frequency of 8 kHz.

\subsubsection{Step-by-Step Analysis}

\begin{enumerate}
    \item \textbf{Zero-Order Hold Impact on Spectrum:}
    A zero-order hold effectively holds each sample value constant over the sample period until the next sample is taken. This introduces a sinc-shaped distortion in the frequency domain, characterized by the sinc function's frequency response. The sinc function causes attenuation and ripple in the frequency spectrum, modifying the original sinewave's amplitude.

    \item \textbf{Baseband Sinewave Analysis:}
    The baseband frequency, as calculated earlier, is 1000 Hz. The power of the sinewave \(P\) is given by \(P = A^2/2\), where \(A\) is the amplitude of the sinewave. For \(x[n]\), \(A = \sqrt{2}\), so \(P = 1\).

    \item \textbf{Power in Decibels (dB) for Baseband Sinewave:}
    Power in decibels is calculated using
    $$
    P_{dB} = 10 \log_{10}(P)
    $$
    Thus, the power in decibels of the baseband sinewave is:
    $$
    P_{dB} = 10 \log_{10}(1) = 0 \text{ dB}
    $$
    This result assumes no gain or loss in the system, which is idealized given the zero-order hold effect, which actually introduces some attenuation due to the sinc function's nature.

    \item \textbf{First Out-of-Band Sinewave Analysis:}
    The first out-of-band frequency component occurs due to the zero-order hold and would typically be at the first zero of the sinc function, which is at the sampling frequency minus the base frequency (8000 Hz - 1000 Hz = 7000 Hz). Due to the nature of the sinc function, the power at the first zero is theoretically zero, but just before this point, there is significant attenuation.

    \item \textbf{Power in Decibels (dB) for the First Out-of-Band Sinewave:}
    Estimating the power just before the zero crossing, considering sinc function's typical roll-off, might give a very low power level. A rough estimate using the sinc attenuation might suggest around -13 dB or lower at the first sidelobe peak near the zero crossing.
\end{enumerate}

\subsubsection{Conclusion}
\begin{itemize}
    \item The power of the baseband sinewave, considering an ideal DAC without accounting for practical attenuation due to ZOH, is 0 dB.
    \item The power of the first out-of-band sinewave would be significantly lower, potentially around -13 dB or more, depending on the exact behavior of the sinc function's first sidelobe.
\end{itemize}

This analysis assumes an idealized scenario and does not factor in some practical aspects like filter design and DAC non-linearities.
