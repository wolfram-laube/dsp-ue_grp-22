\item[(f)]
\section{Derive the Difference Equation}

\subsection*{Problem Statement}
Given the poles and zeros, derive the difference equation from the transfer function \( H(z) \).

\subsection*{Theoretical Background}
The transfer function \( H(z) \) relates the input and output of a system in the z-domain. By expressing \( H(z) \) in terms of its polynomial coefficients, we can convert it back to the time-domain difference equation.

\subsection*{Mathematical Derivation}
The transfer function derived previously is:
\[ H(z) = \frac{z^2}{z^2 + \frac{1}{15} z - \frac{2}{5}} \]

We can express this as:
\[ H(z) = \frac{Y(z)}{X(z)} = \frac{z^2}{z^2 + \frac{1}{15} z - \frac{2}{5}} \]

This gives:
\[ Y(z) \left( z^2 + \frac{1}{15} z - \frac{2}{5} \right) = z^2 X(z) \]

In the time domain, this corresponds to the difference equation:
\[ y[n] + \frac{1}{15} y[n-1] - \frac{2}{5} y[n-2] = x[n] \]

Rearranging terms to isolate \( y[n] \):
\[ y[n] = x[n] - \frac{1}{15} y[n-1] + \frac{2}{5} y[n-2] \]

\subsection*{Conclusion}
The derived difference equation from the transfer function \( H(z) \) is:
\[ y[n] = x[n] - \frac{1}{15} y[n-1] + \frac{2}{5} y[n-2] \]
This confirms the given difference equation.
