\item[(e)]
\section{System Stability}

\subsection*{Problem Statement}
Is the system stable? Explain your answer.

\subsection*{Theoretical Background}
A discrete-time system is stable if its poles lie inside the unit circle in the z-plane. This means that for a system described by its transfer function \( H(z) \), the system is stable if all the poles of \( H(z) \) satisfy \( |z| < 1 \).

\subsection*{Mathematical Derivation}
From Task (d), we have the poles of the transfer function \( H(z) \):
\[ z = \frac{3}{5}, \quad z = -\frac{2}{3} \]

To determine the stability of the system, we need to check if the magnitudes of these poles are less than 1.

\begin{itemize}
    \item For \( z = \frac{3}{5} \):
    \[ \left| \frac{3}{5} \right| = 0.6 < 1 \]
    \item For \( z = -\frac{2}{3} \):
    \[ \left| -\frac{2}{3} \right| = 0.67 < 1 \]
\end{itemize}

Since both poles have magnitudes less than 1, they lie inside the unit circle.

\subsection*{Conclusion}
The system is stable because all the poles of the transfer function \( H(z) \) lie inside the unit circle in the z-plane.
