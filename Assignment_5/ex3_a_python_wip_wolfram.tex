\item[(a)]
\section{Real Coefficients of the Filter}

\subsection*{Problem Statement}
Does this filter have real coefficients? Justify your answer.

Given:
\begin{itemize}
    \item Zeros: \( N_{1}=-1, N_{2}=j, N_{3}=-j \)
    \item Poles: \( P_{1}=0, P_{2}=0.75+j0.25, P_{3}=0.75-j0.25 \)
\end{itemize}

\subsection*{Theoretical Background}
A filter has real coefficients if the poles and zeros either occur in complex conjugate pairs or are real numbers. This ensures that the filter's impulse response is real.

\subsection*{Mathematical Derivation}
The given zeros are:
\begin{itemize}
    \item \( N_{1} = -1 \) (real)
    \item \( N_{2} = j \) (complex)
    \item \( N_{3} = -j \) (complex conjugate of \( j \))
\end{itemize}

The given poles are:
\begin{itemize}
    \item \( P_{1} = 0 \) (real)
    \item \( P_{2} = 0.75 + j0.25 \) (complex)
    \item \( P_{3} = 0.75 - j0.25 \) (complex conjugate of \( 0.75 + j0.25 \))
\end{itemize}

Since the zeros and poles either occur in complex conjugate pairs or are real numbers, the filter has real coefficients.

\subsection*{Conclusion}
The filter has real coefficients because the poles and zeros either occur in complex conjugate pairs or are real numbers.
