\documentclass[12pt,a4paper,austrian]{article}
\usepackage{graphicx}
\usepackage[austrian, english]{babel}
\usepackage[utf8]{inputenc}
\usepackage{listings}
\usepackage{multirow}
\usepackage{epstopdf}
\usepackage{amsmath}
\usepackage{amssymb}
\usepackage{hyperref} % fuer Mengen \N, Q, C, R
\usepackage{minted} % fuer Code Listings mit Syntax Highlighting
\graphicspath{{./fig/}}


%% Satzspiegel
\setlength{\hoffset}{-1in} \setlength{\textwidth}{18cm}
\setlength{\oddsidemargin}{1.5cm}
\setlength{\evensidemargin}{1.5cm}
\setlength{\marginparsep}{0.7em}
\setlength{\marginparwidth}{0.5cm}

\setlength{\voffset}{-1.9in}
\setlength{\headheight}{12pt}
\setlength{\topmargin}{2.6cm}
\addtolength{\topmargin}{-\headheight}
\setlength{\headsep}{3.5cm}
\addtolength{\headsep}{-\topmargin}
\addtolength{\headsep}{-\headheight}
\setlength{\textheight}{27cm}

%% How should floats be treated?
\setlength{\floatsep}{12 pt plus 0 pt minus 8 pt}
\setlength{\textfloatsep}{12 pt plus 0pt minus 8 pt}
\setlength{\intextsep}{12 pt plus 0pt minus 8 pt}

\tolerance2000
\emergencystretch20pt

%% Text appearence
% English text
\newcommand{\eg}[1]%
{\selectlanguage{english}\textit{#1}\selectlanguage{austrian}}

\newcommand{\filename}[1]
{\begin{small}\texttt{#1}\end{small}}

\newcommand\IFT{\unitlength1mm\begin{picture}(10,2) \put (1,1)
{\circle{1.7}} \put(2,1){\line(1,0){5}} \put(8,1)
{\circle*{1.7}}\end{picture}}
\newcommand\FT{\unitlength1mm\begin{picture}(10,2) \put (1,1)
{\circle*{1.7}} \put(2,1){\line(1,0){5}} \put(8,1)
{\circle{1.7}}\end{picture}}

% A box for multiple choice problems
\newcommand{\choicebox}{\fbox{\rule{0pt}{0.5ex}\rule{0.5ex}{0pt}}}

\newenvironment{wahrfalsch}%
{\bigskip\par\noindent\makebox[1cm][c]{richtig}\hspace{3mm}\makebox[1cm][c]{falsch}
    \begin{list}%
    {\makebox[1cm][c]{\choicebox}\hspace{3mm}\makebox[1cm][c]{\choicebox}}%
    {\setlength{\labelwidth}{2.31 cm}\setlength{\labelsep}{3mm}
    \setlength{\leftmargin}{2.61 cm}\setlength{\listparindent}{0pt}
    \setlength{\itemindent}{0pt}}%
    }
    {\end{list}}

\newcounter{theaufgabe}\setcounter{theaufgabe}{1}
\newenvironment{aufgabe}[1]%
{\bigskip\par\noindent\begin{nopagebreak}
                          \textsf{\textbf{\arabic{theaufgabe}.\thinspace Aufgabe}}\quad
                          \textsf{\textit{#1}}\\*[1ex]%
                          \stepcounter{theaufgabe}\hspace{2ex}
\end{nopagebreak}}
{\par\pagebreak[2]}

% Innerhalb der Aufgaben erfolgt die weitere Unterteilung mittels einer
% enumerate Umgebung, die allerdings a), b),... zaehlen soll.
\renewcommand{\labelenumi}{\alph{enumi})}
\renewcommand{\labelenumii}{\arabic{enumii})}

% A box to tick for everything which has to done
\newcommand{\abgabe}{\marginpar{$\Box$}}
% Margin paragraphs on the left side
\reversemarginpar

% Language for listings
%\lstset{language=Vhdl,
%  basicstyle=\small\tt,
% keywordstyle=\tt\bf,
% commentstyle=\sl}

% No indention
\setlength{\parindent}{0.0cm}
% Don't number sections
\setcounter{secnumdepth}{0}


%% Beginning of the text

\begin{document}
    \selectlanguage{austrian}
    \pagestyle{plain}


%===  This is the header section ============================================================
    \thispagestyle{empty}
    \noindent
    \begin{minipage}[b][4cm]{1.0\textwidth}
        \begin{center}
            \begin{bf}
                \begin{large}
                    Digital Signal Processing SS 2024 -- 5.~Assignment
                \end{large} \\
                \vspace{0.3cm}
                \begin{Large}
                    z-Transform, Digital Filters
                \end{Large} \\
                \vspace{0.3cm}
            \end{bf}
            \begin{large}
                Group 22\\
                Julian Feichtinger, K12015812\\
                Wolfram Laube, K08900915\\
            \end{large}
        \end{center}
    \end{minipage}

    \noindent \rule[0.8em]{\textwidth}{0.12mm}\\[-0.5em]
%=======================================================================================


    \begin{aufgabe}{Signal Distortion and Group Delay (20\%)}

        Generate three periods of the signal

        $$
        x[n]=\sum_{i=1}^{4} \frac{1}{2 i-1} \sin (2 \pi 0.005(2 i-1) n)
        $$

        and load (MATLAB command load) the provided mat file \texttt{Filter\_coefficients.mat}.
        This file contains the filter coefficients of an FIR filter (b1 and a1) and the filter coefficients of an IIR filter (b2 and a2).
        Both filters are designed to fulfil the same design criteria (filter specification).
        Filter with both the FIR and the IIR filter the signal $x[n]$ and observe the signal distortion.
        Show and discuss the results in your report.

        \hrule
        \section*{Signal Distortion and Group Delay}

\subsection*{Problem Statement}
Generate three periods of the signal:
\[ x[n]=\sum_{i=1}^{4} \frac{1}{2 i-1} \sin (2 \pi 0.005(2 i-1) n) \]
Load the provided filter coefficients from \texttt{Filter\_coefficients.mat} and filter the signal using both FIR and IIR filters. Observe the signal distortion and discuss the results.

\subsection*{Theoretical Background}
This exercise involves generating a composite signal formed by summing four sinusoidal components with specific frequencies and amplitudes. The signal \( x[n] \) is given by:
\[ x[n]=\sum_{i=1}^{4} \frac{1}{2 i-1} \sin (2 \pi 0.005(2 i-1) n) \]

Filtering this signal using two different types of filters (FIR and IIR) allows us to observe how each filter type affects the signal.

\subsection*{Mathematical Derivation}
The formula for each component of the signal can be derived as:
\[ \text{Component}_i = \frac{1}{2i-1} \sin(2 \pi 0.005 (2i-1) n) \]
The final signal \( x[n] \) is the sum of these four components.

Given the filter coefficients \( b1, a1 \) for the FIR filter and \( b2, a2 \) for the IIR filter, the filtering process can be described as:
\[ y_{\text{FIR}}[n] = \text{filter}(b1, a1, x[n]) \]
\[ y_{\text{IIR}}[n] = \text{filter}(b2, a2, x[n]) \]

\subsection*{Implementation and Results}
The generated signal and the filtered signals are obtained through Python code. The plots below illustrate the generated signal and the filtered signals using both FIR and IIR filters.

\begin{figure}[h]
    \centering
    \includegraphics[width=0.8\textwidth]{fig/ex1_generated_signal.png}
    \caption{Generated Signal \( x[n] \)}
    \label{fig:ex1_generated_signal}
\end{figure}

\begin{figure}[h]
    \centering
    \includegraphics[width=0.8\textwidth]{fig/ex1_filtered_signals.png}
    \caption{Filtered Signals: FIR and IIR}
    \label{fig:ex1_filtered_signals}
\end{figure}

\subsection*{Conclusion}
The FIR and IIR filtered signals show how each filter type affects the original signal. FIR filters tend to introduce a uniform delay while preserving the waveform shape, whereas IIR filters can introduce phase distortions but achieve better magnitude response with a lower order.


    \end{aufgabe}

    \begin{aufgabe}{z-Transform (25\%)}

        Consider the difference equation of a recursive LTI system

        $$
        y[n]=x[n]-\frac{1}{15} y[n-1]+\frac{2}{5} y[n-2]
        $$

        \begin{enumerate}
            \item[(a)] Sketch the block diagram of the LTI system corresponding to the given difference equation.
            \item[(b)] Determine the filter type (FIR or IIR) and explain your choice in the report.
            \item[(c)] Compute the transfer function $H(z)=\frac{Y(z)}{X(z)}$.
            \item[(d)] Calculate the poles and zeros of $H(z)$ and include a sketch of the pole-zero map in your report.
            Also include the region of convergence (ROC) in the pole-zero map.
            \item[(e)] Is the system stable?
            Explain your answer.
        \end{enumerate}

        \hrule

        \begin{enumerate}
            \item[(a)]
\section{Block Diagram of the LTI System}

\subsection*{Problem Statement}
Sketch the block diagram of the LTI system corresponding to the given difference equation:
\[ y[n] = x[n] - \frac{1}{15} y[n-1] + \frac{2}{5} y[n-2] \]

\subsection*{Theoretical Background}
A block diagram for an LTI (Linear Time-Invariant) system visually represents the flow of signals and the operations applied to them. The difference equation can be transformed into a block diagram using delay elements \( z^{-1} \), multipliers, and adders.

\subsection*{Mathematical Derivation}
The given difference equation is:
\[ y[n] = x[n] - \frac{1}{15} y[n-1] + \frac{2}{5} y[n-2] \]

This can be expressed in terms of block diagram components:
\begin{itemize}
    \item \( x[n] \) is the input signal.
    \item \( y[n] \) is the output signal.
    \item \( y[n-1] \) and \( y[n-2] \) are delayed versions of the output signal.
    \item Multipliers \( -\frac{1}{15} \) and \( \frac{2}{5} \) scale the delayed outputs.
    \item An adder combines the scaled delayed outputs with the input signal to produce the output.
\end{itemize}

\subsection*{Block Diagram}
The block diagram representation is as follows:

\begin{figure}[h]
    \centering
    \includegraphics[width=0.8\textwidth]{fig/ex2_a_block_diagram.png}
    \caption{Block Diagram of the LTI System}
    \label{fig:ex2_a_block_diagram}
\end{figure}

\subsection*{Conclusion}
The block diagram illustrates the flow of the input signal \( x[n] \), the delayed outputs \( y[n-1] \) and \( y[n-2] \), and their scaled versions being combined to form the output signal \( y[n] \).

            %! Author = wolfram_e_laube
%! Date = 06.05.24

\item[(b)]
\begin{figure}[h]
    \centering
    \includegraphics[width=0.49\textwidth]{fig/ex2_b_plot}
    \caption{Spectrum of \(x(t)\)}
    \label{fig:ex2_b_plot}
\end{figure}

            \item[(c)]
\section{Transfer Function}

\subsection*{Problem Statement}
Compute the transfer function \( H(z) = \frac{Y(z)}{X(z)} \).

\subsection*{Theoretical Background}
The transfer function \( H(z) \) of a system characterizes its input-output relationship in the z-domain. It is obtained by taking the z-transform of the difference equation and solving for \( H(z) = \frac{Y(z)}{X(z)} \).

\subsection*{Mathematical Derivation}
Given the difference equation:
\[ y[n] = x[n] - \frac{1}{15} y[n-1] + \frac{2}{5} y[n-2] \]

Taking the z-transform of both sides:
\[ Y(z) = X(z) - \frac{1}{15} Y(z) z^{-1} + \frac{2}{5} Y(z) z^{-2} \]

Rearranging to solve for \( \frac{Y(z)}{X(z)} \):
\[ Y(z) + \frac{1}{15} Y(z) z^{-1} - \frac{2}{5} Y(z) z^{-2} = X(z) \]
\[ Y(z) \left( 1 + \frac{1}{15} z^{-1} - \frac{2}{5} z^{-2} \right) = X(z) \]
\[ H(z) = \frac{Y(z)}{X(z)} = \frac{1}{1 + \frac{1}{15} z^{-1} - \frac{2}{5} z^{-2}} \]

Multiplying the numerator and denominator by \( z^2 \) to get a standard form:
\[ H(z) = \frac{z^2}{z^2 + \frac{1}{15} z - \frac{2}{5}} \]

\subsection*{Conclusion}
The transfer function \( H(z) \) of the system is:
\[ H(z) = \frac{z^2}{z^2 + \frac{1}{15} z - \frac{2}{5}} \]
This transfer function characterizes the input-output relationship of the given difference equation in the z-domain.

            \item[(d)]
\section*{Task (d)}

\subsection*{Problem Statement}
What is the frequency spacing between two neighboring spectral points in the DFT spectrum now, after zero-padding to a length of 128 samples?

\subsection*{Theoretical Background}
The frequency resolution of the Discrete Fourier Transform (DFT) is determined by the sampling frequency \( f_s \) and the number of points \( N \) in the DFT. The frequency resolution \( \Delta f \) is given by:

\[ \Delta f = \frac{f_s}{N} \]

Zero-padding a signal to increase its length \( N \) affects the frequency resolution of the DFT. With an increased length, the frequency resolution improves, meaning the spacing between neighboring spectral points becomes finer.

\subsection*{Mathematical Derivation}
Given:
\begin{itemize}
    \item The sampling time \( T_s = 1 \, \text{ms} \)
    \item The new number of points \( N = 128 \)
\end{itemize}

First, calculate the sampling frequency \( f_s \):

\[ f_s = \frac{1}{T_s} = \frac{1}{1 \times 10^{-3} \, \text{s}} = 1000 \, \text{Hz} \]

Next, calculate the new frequency resolution \( \Delta f \):

\[ \Delta f = \frac{f_s}{N} = \frac{1000 \, \text{Hz}}{128} \approx 7.8125 \, \text{Hz} \]

Thus, the frequency spacing between two neighboring spectral points in the DFT spectrum after zero-padding to a length of 128 samples is approximately \( 7.8125 \, \text{Hz} \).

\subsection*{Conclusion}
The frequency resolution of the 128-point DFT/FFT calculated from the zero-padded discrete-time signal \( x[n] \) is approximately \( 7.8125 \, \text{Hz} \). This means that each spectral point in the DFT spectrum is now spaced by \( 7.8125 \, \text{Hz} \), providing a finer frequency resolution compared to the original 100-point DFT.

            \item[(e)]
\section{System Stability}

\subsection*{Problem Statement}
Is the system stable? Explain your answer.

\subsection*{Theoretical Background}
A discrete-time system is stable if its poles lie inside the unit circle in the z-plane. This means that for a system described by its transfer function \( H(z) \), the system is stable if all the poles of \( H(z) \) satisfy \( |z| < 1 \).

\subsection*{Mathematical Derivation}
From Task (d), we have the poles of the transfer function \( H(z) \):
\[ z = \frac{3}{5}, \quad z = -\frac{2}{3} \]

To determine the stability of the system, we need to check if the magnitudes of these poles are less than 1.

\begin{itemize}
    \item For \( z = \frac{3}{5} \):
    \[ \left| \frac{3}{5} \right| = 0.6 < 1 \]
    \item For \( z = -\frac{2}{3} \):
    \[ \left| -\frac{2}{3} \right| = 0.67 < 1 \]
\end{itemize}

Since both poles have magnitudes less than 1, they lie inside the unit circle.

\subsection*{Conclusion}
The system is stable because all the poles of the transfer function \( H(z) \) lie inside the unit circle in the z-plane.

            \item[(f)]
\section{Derive the Difference Equation}

\subsection*{Problem Statement}
Given the poles and zeros, derive the difference equation from the transfer function \( H(z) \).

\subsection*{Theoretical Background}
The transfer function \( H(z) \) relates the input and output of a system in the z-domain. By expressing \( H(z) \) in terms of its polynomial coefficients, we can convert it back to the time-domain difference equation.

\subsection*{Mathematical Derivation}
The transfer function derived previously is:
\[ H(z) = \frac{z^2}{z^2 + \frac{1}{15} z - \frac{2}{5}} \]

We can express this as:
\[ H(z) = \frac{Y(z)}{X(z)} = \frac{z^2}{z^2 + \frac{1}{15} z - \frac{2}{5}} \]

This gives:
\[ Y(z) \left( z^2 + \frac{1}{15} z - \frac{2}{5} \right) = z^2 X(z) \]

In the time domain, this corresponds to the difference equation:
\[ y[n] + \frac{1}{15} y[n-1] - \frac{2}{5} y[n-2] = x[n] \]

Rearranging terms to isolate \( y[n] \):
\[ y[n] = x[n] - \frac{1}{15} y[n-1] + \frac{2}{5} y[n-2] \]

\subsection*{Conclusion}
The derived difference equation from the transfer function \( H(z) \) is:
\[ y[n] = x[n] - \frac{1}{15} y[n-1] + \frac{2}{5} y[n-2] \]
This confirms the given difference equation.

        \end{enumerate}

    \end{aufgabe}

    \begin{aufgabe}{Recursive Filter (25\%)}

        Let the poles and zeros of a recursive filter be given:
        \begin{enumerate}
            \item[-] Zeros: $N_{1}=-1, N_{2}=j, N_{3}=-j$
            \item[-] Poles: $P_{1}=0, P_{2}=0.75+j 0.25, P_{3}=0.75-j 0.25$
        \end{enumerate}

        \begin{enumerate}
            \item[(a)] Does this filter have real coefficients? Justify your answer.
            \item[(b)] Draw a sketch of the pole-zero map.
            \item[(c)] State the transfer function, first with polynomials of $z^{-i}, i=0,1,2,3, \ldots$, and afterwards with polynomials of $z^{+i}$.
            \item[(d)] Draw the block diagram of a direct-form-l implementation of the filter and specify the coefficient values in the block diagram.
            \item[(e)] Plot the magnitude and phase response of the filter in MATLAB.
            \item[(f)] Plot the impulse response of the filter for $0 \leq n \leq 50$ in MATLAB.
        \end{enumerate}

        \hrule

        \begin{enumerate}
            \item[(a)]
\section{Real Coefficients of the Filter}

\subsection*{Problem Statement}
Does this filter have real coefficients? Justify your answer.

Given:
\begin{itemize}
    \item Zeros: \( N_{1}=-1, N_{2}=j, N_{3}=-j \)
    \item Poles: \( P_{1}=0, P_{2}=0.75+j0.25, P_{3}=0.75-j0.25 \)
\end{itemize}

\subsection*{Theoretical Background}
A filter has real coefficients if the poles and zeros either occur in complex conjugate pairs or are real numbers. This ensures that the filter's impulse response is real.

\subsection*{Mathematical Derivation}
The given zeros are:
\begin{itemize}
    \item \( N_{1} = -1 \) (real)
    \item \( N_{2} = j \) (complex)
    \item \( N_{3} = -j \) (complex conjugate of \( j \))
\end{itemize}

The given poles are:
\begin{itemize}
    \item \( P_{1} = 0 \) (real)
    \item \( P_{2} = 0.75 + j0.25 \) (complex)
    \item \( P_{3} = 0.75 - j0.25 \) (complex conjugate of \( 0.75 + j0.25 \))
\end{itemize}

Since the zeros and poles either occur in complex conjugate pairs or are real numbers, the filter has real coefficients.

\subsection*{Conclusion}
The filter has real coefficients because the poles and zeros either occur in complex conjugate pairs or are real numbers.

            %! Author = wolfram_e_laube
%! Date = 16.04.24

\item[(b)]
The Python code to perform the convolution using two nested for-loops is as follows:

\begin{verbatim}
import numpy as np

# Define the impulse response and input signal
h = np.array([0.25, 0.5, 0.25])
x = np.cos(2 * np.pi / 20 * np.arange(50))  # Generate the input signal

# Initialize the output signal array
Ly = len(x) + len(h) - 1
y = np.zeros(Ly)

# Perform convolution using nested for-loops
for n in range(Ly):
    for i in range(len(h)):
        if (n - i >= 0) and (n - i < len(x)):
            y[n] += h[i] * x[n - i]

# Print the output signal
print("Output Signal y[n]:", y)
\end{verbatim}

This Python script manually computes the convolution of the input signal $x[n]$ with the impulse response $h[n]$.
The outer loop increments the output index $n$, while the inner loop increments the memory index $i$.
This method calculates all $L_y$ output samples of $y[n]$ and prints the result.

            \item[(c)]
\section{Transfer Function with Polynomials of \(z^{-i}\) and \(z^{+i}\)}

\subsection*{Problem Statement}
State the transfer function, first with polynomials of \(z^{-i}\), \(i = 0, 1, 2, 3, \ldots\), and afterwards with polynomials of \(z^{+i}\).

\subsection*{Theoretical Background}
The transfer function \(H(z)\) relates the input \(X(z)\) and output \(Y(z)\) in the z-domain. It can be expressed using polynomials of \(z^{-1}\) (commonly used in digital filter design) and also in terms of \(z\).

\subsection*{Mathematical Derivation}
Given the zeros and poles, we can construct the transfer function.

\subsubsection*{Polynomials of \(z^{-i}\)}
The transfer function \( H(z) \) in terms of \( z^{-1} \):
\[ H(z) = \frac{(z + 1)(z - j)(z + j)}{z(z - (0.75 + j0.25))(z - (0.75 - j0.25))} \]

Simplifying the numerator and the denominator:
\[ (z + 1)(z - j)(z + j) = (z + 1)(z^2 + 1) = z^3 + z^2 + z + 1 \]

The denominator:
\[ z(z - (0.75 + j0.25))(z - (0.75 - j0.25)) = z \left(z^2 - (0.75 + j0.25 + 0.75 - j0.25)z + (0.75^2 - j^2(0.25^2))\right) \]
\[ = z(z^2 - 1.5z + 0.625) = z^3 - 1.5z^2 + 0.625z \]

So, the transfer function is:
\[ H(z) = \frac{z^3 + z^2 + z + 1}{z^3 - 1.5z^2 + 0.625z} \]

Expressing in terms of \( z^{-1} \):
\[ H(z) = \frac{1 + z^{-1} + z^{-2} + z^{-3}}{1 - 1.5z^{-1} + 0.625z^{-2}} \]

\subsubsection*{Polynomials of \(z^{+i}\)}
The transfer function \( H(z) \) in terms of \( z \):
\[ H(z) = \frac{z^3 + z^2 + z + 1}{z^3 - 1.5z^2 + 0.625z} \]

\subsection*{Conclusion}
The transfer function expressed with polynomials of \( z^{-i} \) is:
\[ H(z) = \frac{1 + z^{-1} + z^{-2} + z^{-3}}{1 - 1.5z^{-1} + 0.625z^{-2}} \]

The transfer function expressed with polynomials of \( z^{+i} \) is:
\[ H(z) = \frac{z^3 + z^2 + z + 1}{z^3 - 1.5z^2 + 0.625z} \]

            \item[(d)]
\section{Direct-Form I Implementation}

\subsection*{Problem Statement}
Draw the block diagram of a direct-form I implementation of the filter and specify the coefficient values in the block diagram.

\subsection*{Theoretical Background}
A direct-form I implementation of a digital filter uses the difference equation directly to realize the filter structure. It involves using delay elements, multipliers for coefficients, and adders.

Given the transfer function in terms of \( z^{-1} \):
\[ H(z) = \frac{1 + z^{-1} + z^{-2} + z^{-3}}{1 - 1.5z^{-1} + 0.625z^{-2}} \]

The difference equation is:
\[ y[n] = x[n] + x[n-1] + x[n-2] + x[n-3] - 1.5y[n-1] + 0.625y[n-2] \]

\subsection*{Block Diagram}
The block diagram of the direct-form I implementation is shown below:

\begin{figure}[h]
    \centering
    \includegraphics[width=0.8\textwidth]{fig/ex3_d_block_diagram.png}
    \caption{Block Diagram of Direct-Form I Implementation}
    \label{fig:ex3_d_block_diagram}
\end{figure}

\subsection*{Conclusion}
The block diagram of the direct-form I implementation illustrates the structure of the digital filter using delay elements, multipliers, and adders. The coefficient values are specified according to the difference equation.

            \item[(e)]
\section{Magnitude and Phase Response}

\subsection*{Problem Statement}
Plot the magnitude and phase response of the filter.

\subsection*{Theoretical Background}
The frequency response of a digital filter can be analyzed by plotting its magnitude and phase responses. The magnitude response shows how the amplitude of each frequency component is modified by the filter, and the phase response shows the phase shift introduced by the filter at each frequency.

\subsection*{Implementation and Results}
The magnitude and phase response of the filter are computed and plotted using Python. The plots below illustrate the magnitude and phase response of the filter.

\begin{figure}[h]
    \centering
    \includegraphics[width=0.8\textwidth]{fig/ex3_e_magnitude_phase_response.png}
    \caption{Magnitude and Phase Response of the Filter}
    \label{fig:ex3_e_magnitude_phase_response}
\end{figure}

\subsection*{Conclusion}
The magnitude and phase response plots provide insights into how the filter affects different frequency components of the input signal. The magnitude response shows the gain applied to each frequency, and the phase response shows the phase shift introduced by the filter.

            \item[(f)]
\section{Impulse Response}

\subsection*{Problem Statement}
Plot the impulse response of the filter for \(0 \leq n \leq 50\).

\subsection*{Theoretical Background}
The impulse response of a digital filter is the output when the input is an impulse signal (a signal with a value of 1 at \( n = 0 \) and 0 elsewhere). The impulse response characterizes the filter completely in the time domain.

\subsection*{Implementation and Results}
The impulse response of the filter is computed and plotted using Python. The plot below illustrates the impulse response of the filter for \(0 \leq n \leq 50\).

\begin{figure}[h]
    \centering
    \includegraphics[width=0.8\textwidth]{fig/ex3_f_impulse_response.png}
    \caption{Impulse Response of the Filter}
    \label{fig:ex3_f_impulse_response}
\end{figure}

\subsection*{Conclusion}
The impulse response plot provides insights into the time-domain characteristics of the filter. It shows how the filter responds to an impulse input over time.

        \end{enumerate}

    \end{aufgabe}

    \begin{aufgabe}{Lowpass Filter Design (30\%)}
        An analogue signal is sampled with a sampling frequency $f_{\mathrm{s}}=20 \mathrm{kHz}$ and filtered subsequently. The digital lowpass filter should exhibit the following specification:

        \begin{enumerate}
            \item[-] Passband cutoff frequency: $f_{\text {pass }}=3.4 \mathrm{kHz}$
            \item[-] Stopband cutoff frequency: $f_{\text {stop }}=4 \mathrm{kHz}$
            \item[-] Allowed ripple in the passband: $\pm 5 \%$
            \item[-] Minimum stopband attenuation: $45 \mathrm{~dB}$
        \end{enumerate}

        \begin{enumerate}
            \item[(a)]  Specify the normalized radian frequencies for the passband $\Omega_{\text {pass }}$
            and the stopband $\Omega_{\text {stop }}$, the passband tolerance $\delta_{1}$ and the stopband tolerance $\delta_{2}$.
            Hint: Be sure to use the decadic logarithm $\log 10$ ( ) for conversion to decibels.
            \item[(b)]  What is the ideal impulse response $h_{\text {ideal }}[n]$ for the ideal frequency response
            $$
            H_{\text {ideal }}(\Omega)= \begin{cases}
                                            1 & \text { for } 0 \leq|\Omega| \leq \Omega_{0} \\ 0 & \text { for } \Omega_{0} \leq|\Omega| \leq \pi
            \end{cases}
            $$
            with $\Omega_{0}=\frac{\Omega_{\text {pass }}+\Omega_{\text {stop }}}{2}$ ?
            \item[(c)]  Which two measures are necessary to deduce a realizable FIR system of order $N$ from the ideal impulse response $h_{\text {ideal }}[n]$ ?
            \item[(d)]  How can the decrease to the specified filter order $N$ be interpreted, and which effect on the frequency response of the realizable filter does that have?
            \item[(e)]  Design an FIR filter of order $N=20$ with a rectangular window with a corner radian frequency $\Omega_{0}$.
            Plot its frequency response and the tolerance scheme in one plot.
            To this end, complete the provided file \texttt{dsp\_5\_4.m}.
            \item[(f)]  Is the tolerance scheme being violated?
            Can the tolerance scheme be fulfilled by increasing the filter order to $N=90$?
            \item[(g)]  Now, use a hamming window instead of the rectangular window (for $N=90$) and assess the result.
            \item[(h)]  Use the MATLAB filterDesigner to design an elliptic IIR filter fulfilling the above described requirements.
            What is the order of this filter? What is the disadvantage of this filter?
        \end{enumerate}

        \hrule

        \begin{enumerate}
            %! Author = wolfram_e_laube
%! Date = 16.04.24

\item[(a)]
A Python function to calculate the DTFT of a finite sequence is provided below.
The function employs NumPy's vectorized operations to avoid explicit for-loops.

\begin{verbatim}
import numpy as np

def dtft(x, n, w):
    """
    Compute the Discrete-time Fourier Transform (DTFT) of a finite sequence.

    :param x: Finite duration sequence over n (numpy array)
    :param n: Sample position vector (numpy array)
    :param w: Frequency location vector (numpy array)
    :return: DTFT values computed at w frequencies (numpy array)
    """
    # Convert all inputs to numpy arrays to ensure proper calculations
    x = np.array(x)
    n = np.array(n)
    w = np.array(w)

    # Create a 2D meshgrid for the frequencies and samples for broadcasting
    N, W = np.meshgrid(n, w)

    # Compute the DTFT using broadcasting and vectorized operations
    X = np.exp(-1j * N * W) @ x
    return X
\end{verbatim}

This function calculates the DTFT using matrix multiplication, which is a vectorized operation
that can replace explicit looping constructs.

            \item[(b)]
\section*{Task (b)}

\subsection*{Problem Statement}
Compute the spectrum for the whole signal length using the MATLAB command `fft`. Plot the magnitude of the spectrum. Label the axes correctly, with the frequency axis scale in Hz (Hint: the frequency values given in Table 1 should be visible at the correct position on the frequency axis).

\subsection*{Python Script}
\begin{verbatim}
import numpy as np
import matplotlib.pyplot as plt
from scipy.io import wavfile
import os

# Create fig directory if it doesn't exist
if not os.path.exists('fig'):
    os.makedirs('fig')

# Read the DTMF signal from the WAV file
fs, signal = wavfile.read('dtmf.wav')

# Compute the FFT for the whole signal length
spectrum = np.fft.fft(signal)
frequencies = np.fft.fftfreq(len(signal), 1/fs)

# Plot the magnitude spectrum
plt.figure(figsize=(10, 6))
plt.plot(frequencies[:len(frequencies) // 2], np.abs(spectrum[:len(frequencies) // 2]))
plt.title('Magnitude Spectrum of the Entire DTMF Signal')
plt.xlabel('Frequency (Hz)')
plt.ylabel('Magnitude')
plt.grid(True)
plt.savefig('fig/ex4_b_dtmf_spectrum.png')
plt.show()
\end{verbatim}

\subsection*{Magnitude Spectrum of the Entire DTMF Signal}
\begin{figure}[h]
    \centering
    \includegraphics[width=0.8\textwidth]{fig/ex4_b_dtmf_spectrum.png}
    \caption{Magnitude Spectrum of the Entire DTMF Signal}
    \label{fig:ex4_b_dtmf_spectrum}
\end{figure}

\subsection*{Analysis}
The magnitude spectrum of the entire DTMF signal shows the frequencies present in the signal. The frequencies corresponding to the DTMF tones listed in Table 1 should be visible at the correct positions on the frequency axis, confirming the presence of these tones in the signal.

            %! Author = wolfram_e_laube
%! Date = 16.04.24

\item[(c)]
To observe the impact of varying $\Omega$, the Python code below alters the frequency resolution and compares the results:

\begin{verbatim}
import matplotlib.pyplot as plt
import numpy as np

# Varying the frequency resolution
w_coarse = np.linspace(-np.pi, np.pi, 100)
w_fine = np.linspace(-np.pi, np.pi, 1600)
X_coarse = dtft(x, n, w_coarse)
X_fine = dtft(x, n, w_fine)

# Plot the magnitude response for both resolutions
plt.figure(figsize=(14, 5))
plt.subplot(1, 2, 1)
plt.plot(w_coarse, np.abs(X_coarse))
plt.title('Coarse Frequency Resolution')
plt.xlabel('Frequency (rad/sample)')
plt.ylabel('Magnitude')
plt.grid()

plt.subplot(1, 2, 2)
plt.plot(w_fine, np.abs(X_fine))
plt.title('Fine Frequency Resolution')
plt.xlabel('Frequency (rad/sample)')
plt.ylabel('Magnitude')
plt.grid()

plt.tight_layout()
plt.show()
\end{verbatim}

The DTFT results are calculated and plotted with both coarse and fine frequency resolutions to observe differences in the spectrum.
A finer resolution unveils more details in the frequency domain representation of the signal.

            \item[(d)]
\section*{Task (d)}

\subsection*{Problem Statement}
Perform the same steps as in (c), but without multiplying the signal blocks by a Hamming window. How does the resulting magnitude diagram of the STFT differ from the one computed in (c)? How is the effect called that causes this difference?

\subsection*{Python Script}
\begin{verbatim}
import numpy as np
import matplotlib.pyplot as plt
from scipy.io import wavfile
import os

# Create fig directory if it doesn't exist
if not os.path.exists('fig'):
    os.makedirs('fig')

# Read the DTMF signal from the WAV file
fs, signal = wavfile.read('dtmf.wav')

# Parameters for STFT
block_length = 256  # Block length
overlap = block_length // 2  # Overlapping factor of 2

# Pad the signal with zeros if necessary
padding_length = (block_length - len(signal) % block_length) % block_length
padded_signal = np.append(signal, np.zeros(padding_length))

# Generate signal blocks
num_blocks = (len(padded_signal) - overlap) // (block_length - overlap)
blocks = np.zeros((num_blocks, block_length))

for i in range(num_blocks):
    start = i * (block_length - overlap)
    end = start + block_length
    blocks[i, :] = padded_signal[start:end]

# Compute the FFT for each block
ftbs = np.fft.fft(blocks, axis=1)

# Generate the time and frequency vectors for plotting
f_stft = np.fft.fftfreq(block_length, 1/fs)[:block_length // 2]
t_stft = np.arange(num_blocks) * (block_length - overlap) / fs

# Plot the STFT magnitude without Hamming window
plt.figure(figsize=(12, 6))
plt.pcolormesh(t_stft, f_stft, np.abs(ftbs[:, :block_length // 2].T), shading='gouraud')
plt.title('STFT Magnitude of DTMF Signal Without Hamming Window')
plt.xlabel('Time (s)')
plt.ylabel('Frequency (Hz)')
plt.colorbar(label='Magnitude')
plt.ylim(0, 2000)  # Limit the frequency range to 0-2000 Hz
plt.savefig('fig/ex4_d_stft_magnitude_no_hamming.png')
plt.show()
\end{verbatim}

\subsection*{STFT Magnitude of the DTMF Signal Without Hamming Window}
\begin{figure}[h]
    \centering
    \includegraphics[width=0.8\textwidth]{fig/ex4_d_stft_magnitude_no_hamming.png}
    \caption{STFT Magnitude of the DTMF Signal Without Hamming Window}
    \label{fig:ex4_d_stft_magnitude_no_hamming}
\end{figure}

\subsection*{Analysis}
The resulting magnitude diagram of the STFT without the Hamming window shows more spectral leakage compared to the one computed with the Hamming window in Task (c). Spectral leakage occurs because the signal blocks are treated as if they are periodic, leading to discontinuities at the block edges. These discontinuities cause the energy to spread across multiple frequencies, resulting in a less clear representation of the signal's frequency content.

            \item[(e)]
\section{FIR Filter Design with Rectangular Window}

\subsection*{Problem Statement}
Design an FIR filter of order \( N = 20 \) with a rectangular window with a corner radian frequency \( \Omega_0 \). Plot its frequency response and the tolerance scheme in one plot. To this end, complete the provided file `dsp_5_4.m`.

\subsection*{Theoretical Background}
The rectangular window is a simple window function that can be used to truncate the ideal impulse response to a finite length. The frequency response of the resulting FIR filter can then be analyzed and compared with the specified tolerance scheme.

\subsection*{Mathematical Derivation}
The ideal impulse response is given by:
\[ h_{\text{ideal}}[n] = 0.37 \, \text{sinc}(0.37n) \]

To design an FIR filter of order \( N = 20 \) with a rectangular window:
\begin{itemize}
    \item Generate the ideal impulse response \( h_{\text{ideal}}[n] \) for \( n = -10 \) to \( 10 \) (centered around zero).
    \item Apply the rectangular window, which in this case does not change the values of \( h_{\text{ideal}}[n] \) because the window is entirely non-zero over the range.
    \item The resulting windowed and shifted impulse response is the FIR filter coefficients.
\end{itemize}

\subsection*{Implementation and Results}
The frequency response of the designed FIR filter is computed and plotted using Python. The plot below illustrates the frequency response along with the specified tolerance scheme.

\begin{figure}[h]
    \centering
    \includegraphics[width=0.8\textwidth]{fig/ex4_e_frequency_response.png}
    \caption{Frequency Response of FIR Filter with Rectangular Window}
    \label{fig:ex4_e_frequency_response}
\end{figure}

\subsection*{Conclusion}
The frequency response plot shows the performance of the FIR filter designed with a rectangular window of order \( N = 20 \). The tolerance scheme is also illustrated for comparison.

            \item[(f)]
\section{Tolerance Scheme and Filter Order}

\subsection*{Problem Statement}
Is the tolerance scheme being violated? Can the tolerance scheme be fulfilled by increasing the filter order to \( N = 90 \)?

\subsection*{Theoretical Background}
The tolerance scheme specifies the allowable deviations in the passband and stopband of the filter's frequency response. By comparing the frequency response of the designed FIR filter with the specified tolerance scheme, we can determine if the scheme is violated. Increasing the filter order typically results in a sharper transition band and improved adherence to the tolerance scheme.

\subsection*{Analysis}

\subsubsection*{Current Filter Order \( N = 20 \)}
The frequency response of the FIR filter with \( N = 20 \) was previously plotted. By examining this plot, we can check for any violations of the tolerance scheme in the passband ripple and stopband attenuation.

\subsubsection*{Increased Filter Order \( N = 90 \)}
An FIR filter with \( N = 90 \) is designed using the same method (rectangular window). The frequency response is computed and plotted to determine if it meets the tolerance scheme requirements.

\subsection*{Implementation and Results}
The frequency response of the FIR filter with \( N = 90 \) is computed and plotted using Python. The plot below illustrates the frequency response along with the specified tolerance scheme.

\begin{figure}[h]
    \centering
    \includegraphics[width=0.8\textwidth]{fig/ex4_f_frequency_response_90.png}
    \caption{Frequency Response of FIR Filter with Rectangular Window (N=90)}
    \label{fig:ex4_f_frequency_response_90}
\end{figure}

\subsection*{Conclusion}
By comparing the frequency response plots for \( N = 20 \) and \( N = 90 \), we can determine if the tolerance scheme is violated for \( N = 20 \) and if increasing the filter order to \( N = 90 \) fulfills the tolerance scheme requirements.

            \item[(g)]
\section{FIR Filter Design with Hamming Window}

\subsection*{Problem Statement}
Now, use a Hamming window instead of the rectangular window (for \( N = 90 \)) and assess the result.

\subsection*{Theoretical Background}
The Hamming window is another window function used to design FIR filters. It has a better frequency response than the rectangular window, as it reduces the side lobes in the frequency domain, leading to less ripple in the stopband. The Hamming window is defined as:
\[ w[n] = 0.54 - 0.46 \cos \left( \frac{2\pi n}{N-1} \right) \]

\subsection*{Mathematical Derivation}
\begin{enumerate}
    \item Compute the ideal impulse response \( h_{\text{ideal}}[n] \) for \( N = 90 \).
    \item Apply the Hamming window to the ideal impulse response to obtain the windowed impulse response.
    \item Compute the frequency response of the windowed impulse response.
\end{enumerate}

\subsection*{Implementation and Results}
The frequency response of the FIR filter with a Hamming window for \( N = 90 \) is computed and plotted using Python. The plot below illustrates the frequency response along with the specified tolerance scheme.

\begin{figure}[h]
    \centering
    \includegraphics[width=0.8\textwidth]{fig/ex4_g_frequency_response_hamming.png}
    \caption{Frequency Response of FIR Filter with Hamming Window (N=90)}
    \label{fig:ex4_g_frequency_response_hamming}
\end{figure}

\subsection*{Conclusion}
The frequency response plot shows the performance of the FIR filter designed with a Hamming window of order \( N = 90 \). The Hamming window provides a better frequency response compared to the rectangular window, with reduced ripple in the stopband and a sharper transition band.

            \item[(h)]
\section{Elliptic IIR Filter Design}

\subsection*{Problem Statement}
Use the MATLAB filterDesigner to design an elliptic IIR filter fulfilling the above described requirements. What is the order of this filter? What is the disadvantage of this filter?

\subsection*{Theoretical Background}
Elliptic filters, also known as Cauer or Zolotarev filters, provide the steepest transition between passband and stopband for a given filter order. They achieve this by allowing ripple in both the passband and stopband. The disadvantage of elliptic filters is that they introduce more phase distortion compared to other types of filters like Butterworth or Chebyshev filters.

\subsection*{Python Implementation}
The following Python code designs an elliptic IIR filter and plots its frequency response. The filter specifications are:
\begin{itemize}
    \item Sampling frequency: 20 kHz
    \item Passband frequency: 3.4 kHz
    \item Stopband frequency: 4 kHz
    \item Passband ripple: 0.05 (±5%)
    \item Minimum stopband attenuation: 45 dB
\end{itemize}

The frequency response of the designed elliptic IIR filter is shown in the plot below:

\begin{figure}[h]
    \centering
    \includegraphics[width=0.8\textwidth]{fig/ex4_h_frequency_response_elliptic.png}
    \caption{Frequency Response of Elliptic IIR Filter}
    \label{fig:ex4_h_frequency_response_elliptic}
\end{figure}

The filter order calculated by the Python implementation is \textbf{\{N\}} (insert the actual value from the output).

\subsection*{Conclusion}
The elliptic IIR filter designed using the Python `scipy.signal` library fulfills the specified requirements. The main disadvantage of this filter is the increased phase distortion.

        \end{enumerate}

    \end{aufgabe}

\end{document}
