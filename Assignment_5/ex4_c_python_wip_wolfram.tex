\item[(c)]
\section{Measures for Realizable FIR System}

\subsection*{Problem Statement}
Which two measures are necessary to deduce a realizable FIR system of order \( N \) from the ideal impulse response \( h_{\text{ideal}}[n] \)?

\subsection*{Theoretical Background}
The ideal impulse response \( h_{\text{ideal}}[n] \) is non-causal and infinite in length, which makes it non-realizable. To create a realizable FIR filter, two common measures are taken:
\begin{itemize}
    \item \textbf{Windowing}: Applying a window function to truncate the ideal impulse response to a finite length.
    \item \textbf{Shifting}: Shifting the truncated impulse response to make it causal.
\end{itemize}

\subsection*{Mathematical Derivation}

\subsubsection*{Windowing}
The ideal impulse response is infinite. To make it finite, we multiply it by a window function \( w[n] \) that is nonzero only for \( 0 \leq n \leq N-1 \). A common choice is the rectangular window:
\[
w[n] =
\begin{cases}
1 & \text{for } 0 \leq n \leq N-1 \\
0 & \text{otherwise}
\end{cases}
\]
The windowed impulse response \( h_w[n] \) is given by:
\[
h_w[n] = h_{\text{ideal}}[n] \cdot w[n]
\]

\subsubsection*{Shifting}
To make the filter causal, we shift the windowed impulse response by \( \frac{N-1}{2} \) samples to the right:
\[
h_{\text{causal}}[n] = h_w[n - \frac{N-1}{2}]
\]

\subsection*{Conclusion}
The two measures necessary to deduce a realizable FIR system of order \( N \) from the ideal impulse response \( h_{\text{ideal}}[n] \) are:
\begin{itemize}
    \item \textbf{Windowing}: Truncating the ideal impulse response using a window function.
    \item \textbf{Shifting}: Shifting the truncated impulse response to make it causal.
\end{itemize}
