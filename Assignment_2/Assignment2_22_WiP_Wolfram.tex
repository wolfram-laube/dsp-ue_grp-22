\documentclass[12pt,a4paper,austrian]{article}
\usepackage{graphicx}
\usepackage[austrian, english]{babel}
\usepackage[utf8]{inputenc}
%\usepackage{listings}
\usepackage{multirow}
\usepackage{epstopdf}
\usepackage{amsmath}
\usepackage{amssymb} % fuer Mengen \N, Q, C, R
\graphicspath{{./fig/}}


%% Satzspiegel
\setlength{\hoffset}{-1in} \setlength{\textwidth}{18cm}
\setlength{\oddsidemargin}{1.5cm}
\setlength{\evensidemargin}{1.5cm}
\setlength{\marginparsep}{0.7em}
\setlength{\marginparwidth}{0.5cm}

\setlength{\voffset}{-1.9in}
\setlength{\headheight}{12pt}
\setlength{\topmargin}{2.6cm}
   \addtolength{\topmargin}{-\headheight}
\setlength{\headsep}{3.5cm}
   \addtolength{\headsep}{-\topmargin}
   \addtolength{\headsep}{-\headheight}
\setlength{\textheight}{27cm}

%% How should floats be treated?
\setlength{\floatsep}{12 pt plus 0 pt minus 8 pt}
\setlength{\textfloatsep}{12 pt plus 0pt minus 8 pt}
\setlength{\intextsep}{12 pt plus 0pt minus 8 pt}

\tolerance2000
\emergencystretch20pt

%% Text appearence
% English text
\newcommand{\eg}[1]%
  {\selectlanguage{english}\textit{#1}\selectlanguage{austrian}}

\newcommand{\filename}[1]
  {\begin{small}\texttt{#1}\end{small}}

\newcommand\IFT{\unitlength1mm\begin{picture}(10,2) \put (1,1)
{\circle{1.7}} \put(2,1){\line(1,0){5}} \put(8,1)
{\circle*{1.7}}\end{picture}}
\newcommand\FT{\unitlength1mm\begin{picture}(10,2) \put (1,1)
{\circle*{1.7}} \put(2,1){\line(1,0){5}} \put(8,1)
{\circle{1.7}}\end{picture}}

% A box for multiple choice problems
\newcommand{\choicebox}{\fbox{\rule{0pt}{0.5ex}\rule{0.5ex}{0pt}}}

\newenvironment{wahrfalsch}%
  {\bigskip\par\noindent\makebox[1cm][c]{richtig}\hspace{3mm}\makebox[1cm][c]{falsch}
   \begin{list}%
   {\makebox[1cm][c]{\choicebox}\hspace{3mm}\makebox[1cm][c]{\choicebox}}%
   {\setlength{\labelwidth}{2.31 cm}\setlength{\labelsep}{3mm}
    \setlength{\leftmargin}{2.61 cm}\setlength{\listparindent}{0pt}
    \setlength{\itemindent}{0pt}}%
  }
  {\end{list}}

\newcounter{theaufgabe}\setcounter{theaufgabe}{1}
\newenvironment{aufgabe}[1]%
  {\bigskip\par\noindent\begin{nopagebreak}
   \textsf{\textbf{\arabic{theaufgabe}.\thinspace Aufgabe}}\quad
      \textsf{\textit{#1}}\\*[1ex]%
\stepcounter{theaufgabe}\hspace{2ex}\end{nopagebreak}}
  {\par\pagebreak[2]}

% Innerhalb der Aufgaben erfolgt die weitere Unterteilung mittels einer
% enumerate Umgebung, die allerdings a), b),... zaehlen soll.
\renewcommand{\labelenumi}{\alph{enumi})}
\renewcommand{\labelenumii}{\arabic{enumii})}

% A box to tick for everything which has to done
\newcommand{\abgabe}{\marginpar{$\Box$}}
% Margin paragraphs on the left side
\reversemarginpar

% Language for listings
%\lstset{language=Vhdl,
%  basicstyle=\small\tt,
 % keywordstyle=\tt\bf,
 % commentstyle=\sl}

% No indention
\setlength{\parindent}{0.0cm}
% Don't number sections
\setcounter{secnumdepth}{0}


%% Beginning of the text

\begin{document}
\selectlanguage{austrian}
\pagestyle{plain}


%===  This is the header section ============================================================
\thispagestyle{empty}
\noindent
\begin{minipage}[b][4cm]{1.0\textwidth}  
\begin{center}
\begin{bf} 
\begin{large} Digital Signal Processing SS 2024 -- 2.~Assignment\end{large} \\
\vspace{0.3cm}
\begin{Large} Discrete Time Signals, Convolution, DTFT  \end{Large} \\
\vspace{0.3cm}
\end{bf}
\begin{large}
Group 22\\
Julian Feichtinger, K12015812\\
Wolfram Laube, K08900915\\
\end{large} 
\end{center}
\end{minipage}

\noindent \rule[0.8em]{\textwidth}{0.12mm}\\[-0.5em]
%=======================================================================================


\begin{aufgabe}{Discrete Time Signals (30\%)}

\begin{enumerate}

\item[(a)]
The following discrete time signals should be plotted in Matlab as indicated in figure \ref{fig:ex1_required_plot_layout}.
The time axis should be labeled with the correct time values and the signals should be plotted using a solid line.
\[
\begin{array}{ll}
x_{1}[n] = -4 \delta[n+3] + 4 \delta[n] - \delta[n-3] + 2 \delta[n-7] & \text{for } -5 \leq n \leq 10 \\
x_{2}[n] = e^{-0.31 n} & \text{for } -5 \leq n \leq 10 \\
x_{3}[n] = 3 \sin \left(2 \pi \frac{3.5}{64} n\right) & \text{for } 0 \leq n \leq 256 \\
x_{4}[n] = -\cos \left(\frac{9}{64} n\right) & \text{for } 0 \leq n \leq 256
\end{array}
\]
To indicate the discrete time nature of the signals, the \texttt{stem} command should be used, unless the signal length becomes too long, in which case the \texttt{plot} command should be used instead.
Choose the ideal way to plot the signals.
Extend the Matlab script \texttt{dsp\_A2\_1.m}.
Useful commands are \texttt{figure}, \texttt{stem}, \texttt{plot}, \texttt{subplot}, \texttt{xlabel}, \texttt{ylabel}, \texttt{title}, \texttt{legend}, \texttt{hold on}, \texttt{grid on}, \texttt{pi}.
You can use the provided function \texttt{impseq}.

\begin{figure}[h]
\centering
\includegraphics[width=\textwidth]{fig/ex1_required_plot_layout}
\caption{Required arrangement of plots}
\label{fig:ex1_required_plot_layout}
\end{figure}

\item[(b)]
What is the normalized angular frequency $\Omega$ for $x_{3}[n]$ and $x_{4}[n]$?

\item[(c)]
Are the signals $x_{3}[n]$ and $x_{4}[n]$ periodic? If yes, what is their fundamental period?

\item[(d)]
Write the Matlab function \texttt{custom\_power} which calculates the mean power of a discrete time periodic signal according to
\[ P = \frac{1}{N} \sum_{n=0}^{N-1} |x[n]|^2 \]
Pass exactly one signal period to this function to calculate the powers of all periodic signals in (a).

\item[(e)]
Now assume that all signals in (a) are zero outside the plot range (e.g., $x_{2}[11] = 0$).

Write a function \texttt{energy} which calculates the signal energy according to
\[ W = \sum_{n=-\infty}^\infty |x[n]|^2 \]
Determine the energies for all these time-limited signals $x_{1}[n]$ to $x_{4}[n]$.

\item[(f)]
Collect all the signal powers of (d) and signal energies of (e) in a single table in your protocol.
\end{enumerate}

\hrule
\begin{enumerate}
\item[(a)]
\section*{Task (a)}

\subsection*{Problem Statement}
Consider the analog signal
\[ x(t) = 1 + 0.5 \cos \left(2 \pi f_{1} t \right) + 2 \sin \left(2 \pi f_{2} t \right) + \sin \left(2 \pi f_{3} t \right) \]
with \( f_{1} = 2 \, \text{kHz} \), \( f_{2} = 4 \, \text{kHz} \), and \( f_{3} = 6 \, \text{kHz} \).

a) Sketch the Fourier transform of \( x(t) \) and plot the analog signal \( x(t) \) in Python using a time vector \( t = 0:1 \times 10^{-6}:1 \times 10^{-3} \).

\subsection*{Theoretical Background}
The Fourier transform is a mathematical operation that transforms a time-domain signal into its frequency-domain representation. It is crucial for analyzing the frequency components of signals.

Given an analog signal \( x(t) \), its continuous Fourier transform \( X(f) \) is defined as:
\[ X(f) = \int_{-\infty}^{\infty} x(t) e^{-j 2 \pi f t} \, dt \]

For a signal consisting of sinusoids, the Fourier transform will show spikes at the corresponding frequencies.

\subsection*{Mathematical Derivation}
The given signal is:
\[ x(t) = 1 + 0.5 \cos \left(2 \pi f_{1} t \right) + 2 \sin \left(2 \pi f_{2} t \right) + \sin \left(2 \pi f_{3} t \right) \]

The Fourier transform of this signal consists of delta functions at the frequencies of the cosine and sine terms:
\[ X(f) = \delta(f) + 0.25 \left[ \delta(f - f_1) + \delta(f + f_1) \right] - j \left[ \delta(f - f_2) - \delta(f + f_2) \right] - 0.5 j \left[ \delta(f - f_3) - \delta(f + f_3) \right] \]

where:
- The delta function \( \delta(f) \) represents the constant term.
- The cosine term \( 0.5 \cos(2 \pi f_{1} t) \) contributes to delta functions at \( \pm f_1 \).
- The sine terms contribute to delta functions at \( \pm f_2 \) and \( \pm f_3 \), with imaginary coefficients indicating phase shifts.

\subsection*{Python Implementation and Plot}
The plot Figure~\ref{fig:ex1_a_plot} below illustrates the time-domain signal, and Figure~\ref{fig:ex1_a_fft} shows its Fourier transform:

\begin{figure}[h]
    \centering
    \includegraphics[width=0.8\textwidth]{fig/ex1_a_plot.png}
    \caption{Analog Signal $x(t)$}
    \label{fig:ex1_a_plot}
\end{figure}

\begin{figure}[h]
    \centering
    \includegraphics[width=0.8\textwidth]{fig/ex1_a_fft_stem.png}
    \caption{Fourier Transform of $x(t)$}
    \label{fig:ex1_a_fft}
\end{figure}

\subsection*{Conclusion}
The Fourier transform of the given signal shows spikes at the frequencies \( f_1 = 2 \, \text{kHz} \), \( f_2 = 4 \, \text{kHz} \), and \( f_3 = 6 \, \text{kHz} \), corresponding to the cosine and sine terms in the signal. The plot of the analog signal \( x(t) \) illustrates its time-domain behavior over the interval from 0 to 1 ms, and the Fourier transform plot shows the frequency-domain representation with spikes at the expected frequencies.

%! Author = wolfram_e_laube
%! Date = 17.04.24

\item[(b)]
The normalized angular frequency $\Omega$ is calculated as follows:
- For $x_{3}[n] = 3 \sin\left(2 \pi \frac{3.5}{64} n\right)$, $\Omega = \frac{2 \pi \times 3.5}{64}$.
- For $x_{4}[n] = -\cos\left(\frac{9}{64} n\right)$, $\Omega = \frac{2 \pi \times 9}{64}$.

%! Author = wolfram_e_laube
%! Date = 17.04.24

\item[(c)]
Both signals $x_{3}[n]$ and $x_{4}[n]$ are periodic with fundamental periods calculated by $N = \frac{1}{f}$:
- For $x_{3}[n]$, $f = \frac{3.5}{64}$, so $N = \frac{64}{3.5}$.
- For $x_{4}[n]$, $f = \frac{9}{64}$, so $N = \frac{64}{9}$.

%! Author = wolfram_e_laube
%! Date = 17.04.24

\item[(d)]
This function calculates the mean power of a discrete time periodic signal:

\begin{verbatim}
def custom_power(x, N):
    return np.sum(np.abs(x) ** 2) / N
\end{verbatim}

%! Author = wolfram_e_laube
%! Date = 17.04.24

\item[(e)]
This function calculates the total energy of a signal:

\begin{verbatim}
def energy(x):
    return np.sum(np.abs(x) ** 2)
\end{verbatim}

%! Author = wolfram_e_laube
%! Date = 17.04.24

\item[(f)]
All signal powers and energies computed in the previous steps are collected into a single table for a comprehensive overview.
This table facilitates comparison and further analysis:

\begin{verbatim}
import pandas as pd

# Create a dictionary with the data
data = {
    'Signal': ['x1', 'x2', 'x3', 'x4'],
    'Energy': [energy_x1, energy_x2, energy_x3, energy_x4],
    'Power': [None, None, power_x3, power_x4]  # None for non-periodic signals
}

# Create and display a DataFrame
df = pd.DataFrame(data)
print(df)
\end{verbatim}

and reads:

\begin{verbatim}
  Signal       Energy     Power
0     x1    37.000000       NaN
1     x2    48.039373       NaN
2     x3  1152.000000  4.565753
3     x4   129.000000  0.494968
\end{verbatim}

This structure enables easy visualization and access to the energy and power data associated with each signal, making it straightforward to draw comparisons and perform subsequent analyses.

\end{enumerate}

\end{aufgabe}

\begin{aufgabe}{Convolution-1 (20\%)}

The signal $x[n]=(3,-1,2,0,1)$ at sample times $n=(0,1,2,3,4)$ is the input to an LTI system with impulse response $h[n]=(2,3,4,1)$ at sample times $n=(0,1,2,3)$.

\begin{enumerate}
\item[(a)]
How long is the output signal $y[n]$?
\item[(b)]
Manually calculate the output signal $y[n]$.
\end{enumerate}
\hrule

\begin{enumerate}
%! Author = wolfram_e_laube
%! Date = 02.04.24

\item [a)]
Given the cosine wave expressed as
$$
x(t) = \hat{X} \cos(2\pi f_0 t),
$$
we aim to prove its Fourier transform pair. Using Euler's formula, $\cos(\theta) = \frac{e^{j\theta} + e^{-j\theta}}{2}$, we can express the cosine function as a sum of complex exponentials:
$$
x(t) = \hat{X} \left( \frac{e^{j2\pi f_0 t} + e^{-j2\pi f_0 t}}{2} \right) = \frac{\hat{X}}{2} e^{j2\pi f_0 t} + \frac{\hat{X}}{2} e^{-j2\pi f_0 t}.
$$

The Fourier transform of a complex exponential function is given by:
$$
e^{j2\pi f_0 t} \leftrightarrow \delta(f-f_0)
$$
and
$$
e^{-j2\pi f_0 t} \leftrightarrow \delta(f+f_0).
$$

Therefore, applying the Fourier transform to $x(t)$, we obtain:
$$
X(f) = \frac{\hat{X}}{2} \delta(f-f_0) + \frac{\hat{X}}{2} \delta(f+f_0),
$$
proving the given Fourier transform pair for the cosine wave.


%! Author = wolfram_e_laube
%! Date = 16.04.24

\item[(b)]
To manually calculate the output signal $y[n]$, we need to perform the convolution of $x[n]$ and $h[n]$.
This can be expressed as:
$$
y[n] = (x * h)[n] = \sum_{k=-\infty}^\infty x[k] \cdot h[n-k]
$$
However, since $x$ and $h$ are both finite sequences, the summation is only over the overlapping indices:
$$
y[n] = \sum_{k=\max(0, n+1-M)}^{\min(n, N-1)} x[k] \cdot h[n-k]
$$

We'll compute this manually:

\begin{enumerate}
\item[1.] $y[0] = x[0]h[0]$
\item[2.] $y[1] = x[0]h[1] + x[1]h[0]$
\item[3.] $y[2] = x[0]h[2] + x[1]h[1] + x[2]h[0]$
\item[4.] $y[3] = x[0]h[3] + x[1]h[2] + x[2]h[1] + x[3]h[0]$
\item[5.] $y[4] = x[1]h[3] + x[2]h[2] + x[3]h[1] + x[4]h[0]$
\item[6.] $y[5] = x[2]h[3] + x[3]h[2] + x[4]h[1]$
\item[7.] $y[6] = x[3]h[3] + x[4]h[2]$
\item[8.] $y[7] = x[4]h[3]$
\end{enumerate}

The output signal $y[n]$ resulting from the convolution of $x[n]$ and $h[n]$ is given by:
$$
y[n] = (6, 7, 13, 5, 9, 5, 4, 1)
$$

This calculation corresponds to the values derived from manually applying the convolution sum for each output sample.
Each element in $y[n]$ is computed based on the overlapping products of $x[n]$ and $h[n-k]$ for each shift $k$.
\end{enumerate}

\end{aufgabe}

\begin{aufgabe}{Convolution-2 (25\%)}

Consider a Linear Time-Invariant (LTI) system with an impulse response
given by $h[n] = (0.25, 0.5, 0.25)$ at sample indices $n = (0, 1, 2)$.

The input signal is defined as $x[n] = \cos \left(\frac{2 \pi}{20} n\right)$ for $0 \leq n < 50$.

\begin{enumerate}
\item[(a)]
What is the length $L_y$ of the output signal $y[n]$?

\item[(b)]
Implement the convolution operation $y[n] = \sum_{i=0}^{L_{h}-1} h[i] x[n-i]$ in Matlab using two nested for-loops.
The outer for-loop should increment the output index $n$, and the inner for-loop should increment the memory index $i$.
Calculate all $L_y$ output samples of $y[n]$.

\item[(c)]
Plot the signal $y[n]$.

\item[(d)]
Verify the correct implementation of the convolution using the Matlab function \texttt{conv}.
Plot both signals on top of each other, with the directly calculated signal in a solid line style
and the signal obtained via \texttt{conv} in a dashed style.
Provide a legend for your plot.

\end{enumerate}
\hrule

\begin{enumerate}
%! Author = wolfram_e_laube
%! Date = 06.05.24

\item[(a)]
The positive frequencies are \( f_0, 2f_0, \ldots, n f_0 \) where \( f_0 = 1 \mathrm{kHz} \).
%! Author = wolfram_e_laube
%! Date = 16.04.24

\item[(b)]
The Python code to perform the convolution using two nested for-loops is as follows:

\begin{verbatim}
import numpy as np

# Define the impulse response and input signal
h = np.array([0.25, 0.5, 0.25])
x = np.cos(2 * np.pi / 20 * np.arange(50))  # Generate the input signal

# Initialize the output signal array
Ly = len(x) + len(h) - 1
y = np.zeros(Ly)

# Perform convolution using nested for-loops
for n in range(Ly):
    for i in range(len(h)):
        if (n - i >= 0) and (n - i < len(x)):
            y[n] += h[i] * x[n - i]

# Print the output signal
print("Output Signal y[n]:", y)
\end{verbatim}

This Python script manually computes the convolution of the input signal $x[n]$ with the impulse response $h[n]$.
The outer loop increments the output index $n$, while the inner loop increments the memory index $i$.
This method calculates all $L_y$ output samples of $y[n]$ and prints the result.

\item[(c)]
\section{Transfer Function with Polynomials of \(z^{-i}\) and \(z^{+i}\)}

\subsection*{Problem Statement}
State the transfer function, first with polynomials of \(z^{-i}\), \(i = 0, 1, 2, 3, \ldots\), and afterwards with polynomials of \(z^{+i}\).

\subsection*{Theoretical Background}
The transfer function \(H(z)\) relates the input \(X(z)\) and output \(Y(z)\) in the z-domain. It can be expressed using polynomials of \(z^{-1}\) (commonly used in digital filter design) and also in terms of \(z\).

\subsection*{Mathematical Derivation}
Given the zeros and poles, we can construct the transfer function.

\subsubsection*{Polynomials of \(z^{-i}\)}
The transfer function \( H(z) \) in terms of \( z^{-1} \):
\[ H(z) = \frac{(z + 1)(z - j)(z + j)}{z(z - (0.75 + j0.25))(z - (0.75 - j0.25))} \]

Simplifying the numerator and the denominator:
\[ (z + 1)(z - j)(z + j) = (z + 1)(z^2 + 1) = z^3 + z^2 + z + 1 \]

The denominator:
\[ z(z - (0.75 + j0.25))(z - (0.75 - j0.25)) = z \left(z^2 - (0.75 + j0.25 + 0.75 - j0.25)z + (0.75^2 - j^2(0.25^2))\right) \]
\[ = z(z^2 - 1.5z + 0.625) = z^3 - 1.5z^2 + 0.625z \]

So, the transfer function is:
\[ H(z) = \frac{z^3 + z^2 + z + 1}{z^3 - 1.5z^2 + 0.625z} \]

Expressing in terms of \( z^{-1} \):
\[ H(z) = \frac{1 + z^{-1} + z^{-2} + z^{-3}}{1 - 1.5z^{-1} + 0.625z^{-2}} \]

\subsubsection*{Polynomials of \(z^{+i}\)}
The transfer function \( H(z) \) in terms of \( z \):
\[ H(z) = \frac{z^3 + z^2 + z + 1}{z^3 - 1.5z^2 + 0.625z} \]

\subsection*{Conclusion}
The transfer function expressed with polynomials of \( z^{-i} \) is:
\[ H(z) = \frac{1 + z^{-1} + z^{-2} + z^{-3}}{1 - 1.5z^{-1} + 0.625z^{-2}} \]

The transfer function expressed with polynomials of \( z^{+i} \) is:
\[ H(z) = \frac{z^3 + z^2 + z + 1}{z^3 - 1.5z^2 + 0.625z} \]

\item[(d)]
\section{Direct-Form I Implementation}

\subsection*{Problem Statement}
Draw the block diagram of a direct-form I implementation of the filter and specify the coefficient values in the block diagram.

\subsection*{Theoretical Background}
A direct-form I implementation of a digital filter uses the difference equation directly to realize the filter structure. It involves using delay elements, multipliers for coefficients, and adders.

Given the transfer function in terms of \( z^{-1} \):
\[ H(z) = \frac{1 + z^{-1} + z^{-2} + z^{-3}}{1 - 1.5z^{-1} + 0.625z^{-2}} \]

The difference equation is:
\[ y[n] = x[n] + x[n-1] + x[n-2] + x[n-3] - 1.5y[n-1] + 0.625y[n-2] \]

\subsection*{Block Diagram}
The block diagram of the direct-form I implementation is shown below:

\begin{figure}[h]
    \centering
    \includegraphics[width=0.8\textwidth]{fig/ex3_d_block_diagram.png}
    \caption{Block Diagram of Direct-Form I Implementation}
    \label{fig:ex3_d_block_diagram}
\end{figure}

\subsection*{Conclusion}
The block diagram of the direct-form I implementation illustrates the structure of the digital filter using delay elements, multipliers, and adders. The coefficient values are specified according to the difference equation.

\end{enumerate}

\end{aufgabe}

\begin{aufgabe}{Discrete Time Fourier Transform (25\%)}
The time discrete sequence $x[n]=(0.8)^{|n|}(u[n+10]-u[n-11])$ should be transformed to the frequency domain
by using the discrete time Fourier transform (DTFT).
As $x[n]$ has finite support (that means it is a sequence of finite length), $X\left(e^{j \Omega}\right)$
can be determined numerically via Matlab (or alternatively Octave, Python).

\begin{enumerate}
\item[(a)]
Create a Matlab function, that is able to calculate the DTFT of a finite sequence.
The function should provide the following interface:

% Interface is described in the attached screenshot.

Try to avoid for loops!
You are still allowed to use for loops, however there will be a small deduction in points for the case you need to use for loops.

\item[(b)]
Plot magnitude and phase response of $X\left(e^{j \Omega}\right)$ for the interval $-\pi \leq \Omega \leq \pi$
within single subplots.
What do you observe for the phase response (hint: you should have a close look at the scaling of the $y$ axis)?

\item[(c)]
Vary the parameter $\Omega$.
What do you observe in the spectrum?
\end{enumerate}

\hrule

\begin{enumerate}
%! Author = wolfram_e_laube
%! Date = 16.04.24

\item[(a)]
A Python function to calculate the DTFT of a finite sequence is provided below.
The function employs NumPy's vectorized operations to avoid explicit for-loops.

\begin{verbatim}
import numpy as np

def dtft(x, n, w):
    """
    Compute the Discrete-time Fourier Transform (DTFT) of a finite sequence.

    :param x: Finite duration sequence over n (numpy array)
    :param n: Sample position vector (numpy array)
    :param w: Frequency location vector (numpy array)
    :return: DTFT values computed at w frequencies (numpy array)
    """
    # Convert all inputs to numpy arrays to ensure proper calculations
    x = np.array(x)
    n = np.array(n)
    w = np.array(w)

    # Create a 2D meshgrid for the frequencies and samples for broadcasting
    N, W = np.meshgrid(n, w)

    # Compute the DTFT using broadcasting and vectorized operations
    X = np.exp(-1j * N * W) @ x
    return X
\end{verbatim}

This function calculates the DTFT using matrix multiplication, which is a vectorized operation
that can replace explicit looping constructs.

\item[(b)]
\section*{Task (b)}

\subsection*{Problem Statement}
Compute the spectrum for the whole signal length using the MATLAB command `fft`. Plot the magnitude of the spectrum. Label the axes correctly, with the frequency axis scale in Hz (Hint: the frequency values given in Table 1 should be visible at the correct position on the frequency axis).

\subsection*{Python Script}
\begin{verbatim}
import numpy as np
import matplotlib.pyplot as plt
from scipy.io import wavfile
import os

# Create fig directory if it doesn't exist
if not os.path.exists('fig'):
    os.makedirs('fig')

# Read the DTMF signal from the WAV file
fs, signal = wavfile.read('dtmf.wav')

# Compute the FFT for the whole signal length
spectrum = np.fft.fft(signal)
frequencies = np.fft.fftfreq(len(signal), 1/fs)

# Plot the magnitude spectrum
plt.figure(figsize=(10, 6))
plt.plot(frequencies[:len(frequencies) // 2], np.abs(spectrum[:len(frequencies) // 2]))
plt.title('Magnitude Spectrum of the Entire DTMF Signal')
plt.xlabel('Frequency (Hz)')
plt.ylabel('Magnitude')
plt.grid(True)
plt.savefig('fig/ex4_b_dtmf_spectrum.png')
plt.show()
\end{verbatim}

\subsection*{Magnitude Spectrum of the Entire DTMF Signal}
\begin{figure}[h]
    \centering
    \includegraphics[width=0.8\textwidth]{fig/ex4_b_dtmf_spectrum.png}
    \caption{Magnitude Spectrum of the Entire DTMF Signal}
    \label{fig:ex4_b_dtmf_spectrum}
\end{figure}

\subsection*{Analysis}
The magnitude spectrum of the entire DTMF signal shows the frequencies present in the signal. The frequencies corresponding to the DTMF tones listed in Table 1 should be visible at the correct positions on the frequency axis, confirming the presence of these tones in the signal.

%! Author = wolfram_e_laube
%! Date = 16.04.24

\item[(c)]
To observe the impact of varying $\Omega$, the Python code below alters the frequency resolution and compares the results:

\begin{verbatim}
import matplotlib.pyplot as plt
import numpy as np

# Varying the frequency resolution
w_coarse = np.linspace(-np.pi, np.pi, 100)
w_fine = np.linspace(-np.pi, np.pi, 1600)
X_coarse = dtft(x, n, w_coarse)
X_fine = dtft(x, n, w_fine)

# Plot the magnitude response for both resolutions
plt.figure(figsize=(14, 5))
plt.subplot(1, 2, 1)
plt.plot(w_coarse, np.abs(X_coarse))
plt.title('Coarse Frequency Resolution')
plt.xlabel('Frequency (rad/sample)')
plt.ylabel('Magnitude')
plt.grid()

plt.subplot(1, 2, 2)
plt.plot(w_fine, np.abs(X_fine))
plt.title('Fine Frequency Resolution')
plt.xlabel('Frequency (rad/sample)')
plt.ylabel('Magnitude')
plt.grid()

plt.tight_layout()
plt.show()
\end{verbatim}

The DTFT results are calculated and plotted with both coarse and fine frequency resolutions to observe differences in the spectrum.
A finer resolution unveils more details in the frequency domain representation of the signal.

\end{enumerate}

\end{aufgabe}


\end{document}
