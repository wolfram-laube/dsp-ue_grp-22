%! Author = wolfram_e_laube
%! Date = 17.04.24

\item[(d)]
This function calculates the mean power of a discrete time periodic signal over a specified number of samples $N$,
which is typically one period of the signal.
The function is defined to take two arguments: the signal array and its period.

\begin{verbatim}
def custom_power(x, N):
    """
    Calculates the mean power of a discrete time periodic signal over N samples.
    """
    return np.sum(np.abs(x)**2) / N
\end{verbatim}

To calculate the mean power of a periodic signal \(x_3[n]\) and \(x_4[n]\),
where the periods have been previously defined based on their frequencies,
the function is called with the signal truncated to one period:

\begin{verbatim}
# Python example
power_x3 = custom_power(x3_values[:18], 18)
power_x4 = custom_power(x4_values[:7], 7)
\end{verbatim}

and yields:

\begin{verbatim}
Power x_3: 4.5657532755203984
Power x_4: 0.4949681510722214
\end{verbatim}

This approach ensures that the power calculation reflects the actual energy content per cycle of the periodic signals,
providing an accurate measure crucial for signal analysis tasks.