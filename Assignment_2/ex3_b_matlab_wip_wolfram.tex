%! Author = wolfram_e_laube
%! Date = 16.04.24

\item[(b)]
The Python code to perform the convolution using two nested for-loops is as follows:

\begin{verbatim}
import numpy as np

# Define the impulse response and input signal
h = np.array([0.25, 0.5, 0.25])
x = np.cos(2 * np.pi / 20 * np.arange(50))  # Generate the input signal

# Initialize the output signal array
Ly = len(x) + len(h) - 1
y = np.zeros(Ly)

# Perform convolution using nested for-loops
for n in range(Ly):
    for i in range(len(h)):
        if (n - i >= 0) and (n - i < len(x)):
            y[n] += h[i] * x[n - i]

# Print the output signal
print("Output Signal y[n]:", y)
\end{verbatim}

This Python script manually computes the convolution of the input signal $x[n]$ with the impulse response $h[n]$.
The outer loop increments the output index $n$, while the inner loop increments the memory index $i$.
This method calculates all $L_y$ output samples of $y[n]$ and prints the result.
