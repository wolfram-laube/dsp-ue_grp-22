%! Author = wolfram_e_laube
%! Date = 16.04.24

\item[(c)]
To observe the effects of varying the parameter $\Omega$ in the spectrum, the MATLAB code can alter the frequency resolution. Here is an example of how the MATLAB code can be used to compare different frequency resolutions:

\begin{verbatim}
% Different ranges of frequency vector w for observation
w1 = -pi:0.1:pi; % Coarser frequency resolution
w2 = -pi:0.01:pi; % Finer frequency resolution

% Calculate the DTFT for coarser frequency resolution
X1 = dtft(x, n, w1);

% Calculate the DTFT for finer frequency resolution
X2 = dtft(x, n, w2);

% Plot the magnitude and phase response for coarser frequency resolution
figure;

subplot(2,1,1);
plot(w1, abs(X1));
title('Magnitude Response with Coarser $\Omega$ Resolution');
xlabel('$\Omega$');
ylabel('|X|');
grid on;

subplot(2,1,2);
plot(w1, angle(X1));
title('Phase Response with Coarser $\Omega$ Resolution');
xlabel('$\Omega$');
ylabel('Phase (radians)');
grid on;

% Plot the magnitude and phase response for finer frequency resolution
figure;

subplot(2,1,1);
plot(w2, abs(X2));
title('Magnitude Response with Finer $\Omega$ Resolution');
xlabel('$\Omega$');
ylabel('|X|');
grid on;

subplot(2,1,2);
plot(w2, angle(X2));
title('Phase Response with Finer $\Omega$ Resolution');
xlabel('$\Omega$');
ylabel('Phase (radians)');
grid on;
\end{verbatim}

When varying $\Omega$, different frequency resolutions will affect the DTFT computation.
A coarser frequency resolution might overlook some subtle details in the spectrum,
whereas a finer resolution could reveal a more detailed structure in the signal's frequency domain representation.
These observations can be made by comparing the DTFT results obtained using different resolutions of the frequency vector $w$.
