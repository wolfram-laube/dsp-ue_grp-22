%! Author = wolfram_e_laube
%! Date = 16.04.24

\item[(b)]
The magnitude and phase response of $X\left(e^{j \Omega}\right)$ are plotted using the following MATLAB code:

\begin{verbatim}
% Define the sequence parameters
n = -10:10; % Sequence is nonzero between n = -10 and n = 10
x = (0.8).^abs(n) .* (heaviside(n + 10) - heaviside(n - 11));

% Define the frequency vector
w = -pi:0.01:pi; % Frequency range

% Calculate the DTFT using the previously defined dtft function
X = dtft(x, n, w);

% Plot the magnitude and phase response
figure;

% Magnitude response subplot
subplot(2,1,1);
plot(w, abs(X));
title('Magnitude Response of $X(e^{j\Omega})$');
xlabel('$\Omega$');
ylabel('|X|');
grid on;

% Phase response subplot
subplot(2,1,2);
plot(w, angle(X));
title('Phase Response of $X(e^{j\Omega})$');
xlabel('$\Omega$');
ylabel('Phase (radians)');
grid on;
\end{verbatim}

Observations for the magnitude response and phase response are as follows: the magnitude response reveals
the amplitude spectrum of the signal, while the phase response may show piecewise linearity.
The scale of the y-axis in the phase plot should be adjusted to properly observe the phase variations.
