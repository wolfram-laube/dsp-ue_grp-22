%! Author = wolfram_e_laube
%! Date = 16.04.24

\item[(a)]
A Python function to calculate the DTFT of a finite sequence is provided below.
The function employs NumPy's vectorized operations to avoid explicit for-loops.

\begin{verbatim}
import numpy as np

def dtft(x, n, w):
    """
    Compute the Discrete-time Fourier Transform (DTFT) of a finite sequence.

    :param x: Finite duration sequence over n (numpy array)
    :param n: Sample position vector (numpy array)
    :param w: Frequency location vector (numpy array)
    :return: DTFT values computed at w frequencies (numpy array)
    """
    # Convert all inputs to numpy arrays to ensure proper calculations
    x = np.array(x)
    n = np.array(n)
    w = np.array(w)

    # Create a 2D meshgrid for the frequencies and samples for broadcasting
    N, W = np.meshgrid(n, w)

    # Compute the DTFT using broadcasting and vectorized operations
    X = np.exp(-1j * N * W) @ x
    return X
\end{verbatim}

This function calculates the DTFT using matrix multiplication, which is a vectorized operation
that can replace explicit looping constructs.
