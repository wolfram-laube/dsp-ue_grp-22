%! Author = wolfram_e_laube
%! Date = 17.04.24

\item[(a)]
The discrete time signals are plotted with the appropriate commands to display the nature of the signals.
For the signals $x_{1}[n]$ and $x_{2}[n]$, the stem command is used to emphasize their discrete nature,
and for the longer signals $x_{3}[n]$ and $x_{4}[n]$, the plot command is utilized.

\begin{verbatim}
% Define the sample ranges
n1 = -5:10;
n2 = 0:256;

% Define the signals
x1 = -4 * (n1 == -3) + 4 * (n1 == 0) - (n1 == 3) + 2 * (n1 == 7);
x2 = exp(-0.31 * n1);
x3 = 3 * sin(2 * pi * 3.5/64 * n2);
x4 = -cos(9/64 * n2);

% Plot the signals
figure;
subplot(2,2,1);
stem(n1, x1, 'filled');
title('Signal x1[n]');

subplot(2,2,2);
stem(n1, x2, 'filled');
title('Signal x2[n]');

subplot(2,2,3);
plot(n2, x3);
title('Signal x3[n]');

subplot(2,2,4);
plot(n2, x4);
title('Signal x4[n]');
\end{verbatim}
